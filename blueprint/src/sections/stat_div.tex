
\section{Statistical divergences}

\begin{definition}
  \label{def:statInfo}
  \lean{ProbabilityTheory.statInfo}
  \leanok
  \uses{def:bayesBinaryRisk, def:riskIncrease}
  The statistical information between measures $\mu$ and $\nu$ with respect to prior $\xi \in \mathcal M(\{0,1\})$ is
  $\mathcal I_\xi(\mu, \nu) = \min\{\xi_0 \mu(\mathcal X), \xi_1 \nu(\mathcal X)\} - \mathcal B_\xi(\mu, \nu)$.
  This is the risk increase $I_\xi^P(d_{\mathcal X})$ in the binary hypothesis testing problem for $d_{\mathcal X} : \mathcal X \rightsquigarrow *$ the Markov kernel to the point space.
\end{definition}

The statistical information is the difference between the minimal risk of an estimator that does not depend on the data and that of a Bayes estimator.
This is a simple generalization of both the DeGroot statistical information as well as the hockey-stick (or $E_\gamma$) divergence.

\begin{lemma}
  \label{lem:statInfo_self}
  \lean{ProbabilityTheory.statInfo_self}
  \leanok
  \uses{def:statInfo}
  For $\mu \in \mathcal M(\mathcal X)$, $\mathcal I_\xi(\mu, \mu) = 0$~.
\end{lemma}

\begin{proof}\leanok
\uses{lem:bayesBinaryRisk_self}
Use Lemma~\ref{lem:bayesBinaryRisk_self}.
\end{proof}

\begin{lemma}
  \label{lem:statInfo_nonneg}
  %\lean{}
  %\leanok
  \uses{def:statInfo}
  For $\mu, \nu \in \mathcal M(\mathcal X)$, $\mathcal I_\xi(\mu, \nu) \ge 0$~.
\end{lemma}

\begin{proof}%\leanok
\uses{lem:bayesBinaryRisk_le}
Use Lemma~\ref{lem:bayesBinaryRisk_le}.
\end{proof}


\begin{lemma}
  \label{lem:statInfo_le}
  \lean{ProbabilityTheory.statInfo_le_min}
  \leanok
  \uses{def:statInfo}
  For $\mu, \nu \in \mathcal M(\mathcal X)$, $\mathcal I_\xi(\mu, \nu) \le \min\{\xi_0 \mu(\mathcal X), \xi_1 \nu(\mathcal X)\}$~.
\end{lemma}

\begin{proof}\leanok
\uses{lem:bayesBinaryRisk_nonneg}
Use Lemma~\ref{lem:bayesBinaryRisk_nonneg}.
\end{proof}


\begin{lemma}
  \label{lem:statInfo_symm}
  \lean{ProbabilityTheory.statInfo_symm}
  \leanok
  \uses{def:statInfo}
  For $\mu, \nu \in \mathcal M(\mathcal X)$ and $\xi \in \mathcal M(\{0,1\})$, $\mathcal I_\xi(\mu, \nu) = \mathcal I_{\xi_\leftrightarrow}(\nu, \mu)$~.
\end{lemma}

\begin{proof}\leanok
\uses{lem:bayesBinaryRisk_symm}
Use Lemma~\ref{lem:bayesBinaryRisk_symm}.
\end{proof}

\begin{theorem}[Data-processing inequality]
  \label{thm:data_proc_statInfo}
  \lean{ProbabilityTheory.statInfo_comp_le}
  \leanok
  \uses{def:statInfo}
  For $\mu, \nu \in \mathcal M(\mathcal X)$, $\xi \in \mathcal M(\{0,1\})$ and $\kappa : \mathcal X \rightsquigarrow \mathcal Y$ a Markov kernel, $\mathcal I_\xi(\kappa \circ \mu, \kappa \circ \nu) \le \mathcal I_\xi(\mu, \nu)$~.
\end{theorem}

\begin{proof}\leanok
\uses{thm:data_proc_bayesBinaryRisk, lem:riskIncrease_comp_del}
Since $\kappa$ is a Markov kernel,
\begin{align*}
\mathcal I_\xi(\kappa \circ \mu, \kappa \circ \nu)
&= \min\{\xi_0(\kappa \circ \mu)(\mathcal X), \xi_1(\kappa \circ \nu)(\mathcal X)\} - \mathcal B_\xi(\kappa \circ \mu, \kappa \circ \nu)
\\
&= \min\{\xi_0 \mu(\mathcal X), \xi_1 \nu(\mathcal X)\} - \mathcal B_\xi(\kappa \circ \mu, \kappa \circ \nu)
\end{align*}
It suffices to prove $\mathcal B_\xi(\kappa \circ \mu, \kappa \circ \nu) \ge \mathcal B_\xi(\mu, \nu)$ since $\mathcal I_\xi(\mu, \nu) = \min\{\xi_0\mu(\mathcal X), \xi_1\nu(\mathcal X)\} - \mathcal B_\xi(\mu, \nu)$.
The inequality was proved in Theorem~\ref{thm:data_proc_bayesBinaryRisk}.

Alternatively, we can use Lemma~\ref{lem:riskIncrease_comp_del}.
\end{proof}


\begin{corollary}
  \label{cor:statInfo_data_proc_event}
  \lean{ProbabilityTheory.statInfo_boolMeasure_le_statInfo}
  \leanok
  \uses{def:statInfo}
  Let $\mu, \nu$ be two measures on $\mathcal X$, $\xi \in \mathcal M(\{0,1\})$ and let $E$ be an event on $\mathcal X$. Let $\mu_E$ and $\nu_E$ be the two Bernoulli distributions with respective means $\mu(E)$ and $\nu(E)$.
  Then $\mathcal I_\xi(\mu, \nu) \ge \mathcal I_\xi(\mu_E, \nu_E)$.
\end{corollary}

\begin{proof}\leanok
\uses{thm:data_proc_statInfo}
Apply Theorem~\ref{thm:data_proc_statInfo} to the deterministic kernel $\mathcal X \rightsquigarrow \{0,1\}$ given by the function $x \mapsto \mathbb{I}\{x \in E\}$.
\end{proof}


\begin{lemma}
  \label{lem:statInfo_eq_sub_min}
  \lean{ProbabilityTheory.statInfo_eq_min_sub_lintegral, ProbabilityTheory.statInfo_eq_min_sub_lintegral', ProbabilityTheory.toReal_statInfo_eq_min_sub_integral}
  \leanok
  \uses{def:statInfo}
  For finite measures $\mu, \nu$ and $\xi \in \mathcal M(\{0,1\})$, for any measure $\zeta$ with $\mu \ll \zeta$ and $\nu \ll \zeta$~,
  \begin{align*}
  \mathcal I_\xi(\mu, \nu)
  &= \min\{\xi_0\mu(\mathcal X), \xi_1\nu(\mathcal X)\} - \zeta\left[x \mapsto \min \left\{\xi_0\frac{d \mu}{d\zeta}(x), \xi_1\frac{d \nu}{d\zeta}(x)\right\}\right]
  \: .
  \end{align*}
  This holds in particular for $\zeta = P \circ \xi$.
\end{lemma}

\begin{proof}\leanok
\uses{thm:bayesBinaryRisk_eq}
Apply Theorem~\ref{thm:bayesBinaryRisk_eq} to get that result for $\zeta = P \circ \xi$. Then for other $\zeta$ with $\mu \ll \zeta$ and $\nu \ll \zeta$~, use $\frac{d \mu}{d\zeta} = \frac{d \mu}{d(P \circ \xi)} / \frac{d (P \circ \xi)}{d\zeta}$~.
\end{proof}


\begin{lemma}
  \label{lem:statInfo_eq_integral_abs_sub}
  \lean{ProbabilityTheory.statInfo_eq_abs_add_lintegral_abs}
  \leanok
  \uses{def:statInfo}
  For finite measures $\mu, \nu$ and $\xi \in \mathcal M(\{0,1\})$, for any measure $\zeta$ with $\mu \ll \zeta$ and $\nu \ll \zeta$~,
  \begin{align*}
  \mathcal I_\xi(\mu, \nu)
  &= - \frac{1}{2} \left\vert \xi_0\mu(\mathcal X) - \xi_1\nu(\mathcal X) \right\vert + 
  \frac{1}{2} \zeta\left[x \mapsto \left\vert \xi_0\frac{d \mu}{d\zeta}(x) - \xi_1\frac{d \nu}{d\zeta}(x)\right\vert\right] 
  \: .
  \end{align*}
  This holds in particular for $\zeta = P \circ \xi$.
\end{lemma}

\begin{proof}%\leanok
\uses{cor:bayesBinaryRisk_eq_abs}
Apply Corollary~\ref{cor:bayesBinaryRisk_eq_abs} and write $\min\{\xi_0\mu(\mathcal X), \xi_1\nu(\mathcal X)\} = \frac{1}{2}\left((P \circ \xi)(\mathcal X) - \left\vert \xi_0\mu(\mathcal X) - \xi_1\nu(\mathcal X) \right\vert \right)$ to get the result for $\zeta = P \circ \xi$.
The result for other $\zeta$ follows from simple manipulations of Radon-Nikodym derivatives.
\end{proof}


\begin{theorem}[Integral form of the statistical information]
  \label{thm:statInfo_eq_integral}
  \lean{ProbabilityTheory.toReal_statInfo_eq_integral_max_of_le, ProbabilityTheory.toReal_statInfo_eq_integral_max_of_gt}
  \leanok
  \uses{def:statInfo}
  For finite measures $\mu, \nu$ and $\xi \in \mathcal M(\{0,1\})$,
  \begin{align*}
  \mathcal I_\xi(\mu, \nu)
  &= \nu\left[ x \mapsto \max \left\{0 , \xi_0\frac{d \mu}{d\nu}(x) - \xi_1 \right\} \right] + \xi_0 \mu_{\perp \nu}(\mathcal X) & \text{ if } \xi_0 \mu(\mathcal X) \le \xi_1 \nu(\mathcal X)
  \: , \\
  \mathcal I_\xi(\mu, \nu)
  &= \nu\left[ x \mapsto \max \left\{0 , \xi_1 - \xi_0\frac{d \mu}{d\nu}(x) \right\} \right] & \text{ if } \xi_0 \mu(\mathcal X) \ge \xi_1 \nu(\mathcal X)
  \: .
  \end{align*}
  
\end{theorem}

For probability measures, this theorem makes the statistical information an $f$-divergence (which is defined later in this document).

\begin{proof}\leanok
\uses{thm:bayesBinaryRisk_eq}
By Theorem~\ref{thm:bayesBinaryRisk_eq},
\begin{align*}
\mathcal I_\xi(\mu, \nu)
&= \min\{\xi_0\mu(\mathcal X), \xi_1\nu(\mathcal X)\} - (P \circ \xi)\left[x \mapsto \min \left\{\xi_0\frac{d \mu}{d(P \circ \xi)}(x), \xi_1\frac{d \nu}{d(P \circ \xi)}(x)\right\}\right]
\: .
\end{align*}
Suppose that $\xi_0\mu(\mathcal X) \le \xi_1\nu(\mathcal X)$~. Then
\begin{align*}
\mathcal I_\xi(\mu, \nu)
&= \xi_0\mu(\mathcal X) - (P \circ \xi)\left[x \mapsto \xi_0\frac{d \mu}{d(P \circ \xi)}(x) + \min \left\{0 , \xi_1\frac{d \nu}{d(P \circ \xi)}(x) - \xi_0\frac{d \mu}{d(P \circ \xi)}(x)\right\}\right]
\\
&= - (P \circ \xi)\left[x \mapsto \min \left\{0 , \xi_1\frac{d \nu}{d(P \circ \xi)}(x) - \xi_0\frac{d \mu}{d(P \circ \xi)}(x) \right\}\right]
\\
&= (P \circ \xi)\left[x \mapsto \max \left\{0 , \xi_0\frac{d \mu}{d(P \circ \xi)}(x) - \xi_1\frac{d \nu}{d(P \circ \xi)}(x) \right\}\right]
\\
&= \nu\left[ x \mapsto \max \left\{0 , \xi_0\frac{d \mu}{d\nu}(x) - \xi_1 \right\} \right] + \xi_0 \mu_{\perp \nu}(\mathcal X)
\: .
\end{align*}
If on the other hand $\xi_0\mu(\mathcal X) \ge \xi_1\nu(\mathcal X)$~, with similar computations,
\begin{align*}
\mathcal I_\xi(\mu, \nu)
&= (P \circ \xi)\left[x \mapsto \max \left\{0 , \xi_1\frac{d \nu}{d(P \circ \xi)}(x) - \xi_0\frac{d \mu}{d(P \circ \xi)}(x) \right\}\right]
\\
&= \nu\left[ x \mapsto \max \left\{0 , \xi_1 - \xi_0\frac{d \mu}{d\nu}(x) \right\} \right]
\: .
\end{align*}
\end{proof}

\begin{corollary}
  \label{cor:statInfo_eq_integral_abs}
  %\lean{}
  %\leanok
  \uses{def:statInfo}
  For finite measures $\mu, \nu$ and $\xi \in \mathcal M(\{0,1\})$,
  \begin{align*}
  \mathcal I_\xi(\mu, \nu)
  &= -\frac{1}{2} \left\vert\xi_0 \mu(\mathcal X) - \xi_1 \nu(\mathcal X)\right\vert + \frac{1}{2}\left( \nu\left[ x \mapsto \left\vert\xi_0\frac{d \mu}{d\nu}(x) - \xi_1 \right\vert \right]  + \xi_0 \mu_{\perp \nu}(\mathcal X)\right)
  \: .
  \end{align*}
\end{corollary}

\begin{proof}%\leanok
\uses{thm:statInfo_eq_integral}
Start from Theorem~\ref{thm:statInfo_eq_integral}, then use $\max\{a,b\} = \frac{1}{2}\left( a + b + \vert a - b \vert \right)$~.
The two cases of Theorem~\ref{thm:statInfo_eq_integral} lead to the same expression.
\end{proof}


\begin{lemma}
  \label{lem:statInfo_eq_sub_inf_event}
  %\lean{}
  %\leanok
  \uses{def:statInfo}
  For finite measures $\mu, \nu$ and $\xi \in \mathcal M(\{0,1\})$,
  \begin{align*}
  \mathcal I_\xi(\mu, \nu)
  &= \min\{\xi_0 \mu(\mathcal X), \xi_1 \nu(\mathcal X)\} - \inf_{E \text{ event}} \left( \xi_0 \mu(E) + \xi_1 \nu(E^c) \right)
  \: .
  \end{align*}
\end{lemma}

\begin{proof}%\leanok
\uses{lem:bayesBinaryRisk_eq_event}
This is a direct application of Lemma~\ref{lem:bayesBinaryRisk_eq_event}.
\end{proof}


\begin{lemma}
  \label{lem:statInfo_eq_sup_event}
  %\lean{}
  %\leanok
  \uses{def:statInfo}
  For finite measures $\mu, \nu$ and $\xi \in \mathcal M(\{0,1\})$,
  \begin{align*}
  \mathcal I_\xi(\mu, \nu)
  &= - \max\{0, \xi_1 \nu(\mathcal X) - \xi_0 \mu(\mathcal X) \} + \sup_{E \text{ event}} \left( \xi_1 \nu(E) - \xi_0 \mu(E) \right)
  \\
  &= - \max\{0, \xi_0 \mu(\mathcal X) - \xi_1 \nu(\mathcal X) \} + \sup_{E \text{ event}} \left( \xi_0 \mu(E) - \xi_1 \nu(E) \right)
  \: .
  \end{align*}
\end{lemma}

\begin{proof}%\leanok
\uses{lem:statInfo_eq_sub_inf_event}

\end{proof}



\subsection{DeGroot statistical information}

TODO: we could also call the more general statistical information defined above the DeGroot statistical information.

\begin{definition}
  \label{def:deGrootInfo}
  \lean{ProbabilityTheory.deGrootInfo}
  \leanok
  \uses{def:statInfo}
  The DeGroot statistical information between finite measures $\mu$ and $\nu$ for $\pi \in [0,1]$ is $I_\pi(\mu, \nu) = \mathcal I_{(\pi, 1 - \pi)}(\mu, \nu)$~.
\end{definition}

\begin{lemma}
  \label{lem:deGrootInfo_self}
  \lean{ProbabilityTheory.deGrootInfo}
  \leanok
  \uses{def:deGrootInfo}
  For $\mu \in \mathcal M(\mathcal X)$, $I_\pi(\mu, \mu) = 0$~.
\end{lemma}

\begin{proof}%\leanok
\uses{lem:statInfo_self}
Use Lemma~\ref{lem:statInfo_self}.
\end{proof}

\begin{lemma}
  \label{lem:deGrootInfo_nonneg}
  %\lean{}
  %\leanok
  \uses{def:deGrootInfo}
  For $\mu, \nu \in \mathcal M(\mathcal X)$, $I_\pi(\mu, \nu) \ge 0$~.
\end{lemma}

\begin{proof}%\leanok
\uses{lem:statInfo_nonneg}
Use Lemma~\ref{lem:statInfo_nonneg}.
\end{proof}

\begin{lemma}
  \label{lem:deGrootInfo_symm}
  %\lean{}
  %\leanok
  \uses{def:deGrootInfo}
  For $\mu, \nu \in \mathcal M(\mathcal X)$ and $\pi \in [0,1]$, $I_\pi(\mu, \nu) = I_{1 - \pi}(\nu, \mu)$~.
\end{lemma}

\begin{proof}%\leanok
\uses{lem:statInfo_symm}
Use Lemma~\ref{lem:statInfo_symm}.
\end{proof}

\begin{theorem}[Data-processing inequality]
  \label{thm:data_proc_deGrootInfo}
  \lean{ProbabilityTheory.deGrootInfo_comp_le}
  \leanok
  \uses{def:deGrootInfo}
  For $\mu, \nu \in \mathcal M(\mathcal X)$ and $\kappa : \mathcal X \rightsquigarrow \mathcal Y$ a Markov kernel, $I_\pi(\kappa \circ \mu, \kappa \circ \nu) \le I_\pi(\mu, \nu)$~.
\end{theorem}

\begin{proof}\leanok
\uses{thm:data_proc_statInfo}
Apply Theorem~\ref{thm:data_proc_statInfo}.
\end{proof}


\begin{lemma}[Integral form of the DeGroot statistical information]
  \label{lem:deGrootInfo_eq_integral}
  %\lean{}
  %\leanok
  \uses{def:deGrootInfo}
  For finite measures $\mu, \nu$ and $\pi \in [0,1]$,
  \begin{align*}
  I_\pi(\mu, \nu)
  &= \nu\left[ x \mapsto \pi \max \left\{0 , \frac{d \mu}{d\nu}(x) - \frac{1 - \pi}{\pi} \right\} \right] + \pi \mu_{\perp \nu}(\mathcal X) & \text{ if } \pi \mu(\mathcal X) \le (1 - \pi) \nu(\mathcal X)
  \: , \\
  I_\pi(\mu, \nu)
  &= \nu\left[ x \mapsto \pi \max \left\{0 , \frac{1 - \pi}{\pi} - \frac{d \mu}{d\nu}(x) \right\} \right] & \text{ if } \pi \mu(\mathcal X) \ge (1 - \pi) \nu(\mathcal X)
  \: .
  \end{align*}
\end{lemma}

\begin{proof}%\leanok
\uses{thm:statInfo_eq_integral}
Apply Theorem~\ref{thm:statInfo_eq_integral}.
\end{proof}




\subsection{Hockey-stick divergence}


\begin{definition}
  \label{def:eGamma}
  \lean{ProbabilityTheory.eGamma}
  \leanok
  \uses{def:statInfo}
  The $E_\gamma$ or hockey-stick divergence between finite measures $\mu$ and $\nu$ for $\gamma \in (0,+\infty)$ is $E_\gamma(\mu, \nu) = \mathcal I_{(1,\gamma)}(\mu, \nu)$~.
\end{definition}

Note that our definition of the $E_\gamma$ divergence extends the one from the literature: the divergence is usually defined for $\gamma \ge 1$ only.
The extension makes sense in light of Theorem~\ref{thm:fDiv_eq_integral_eGamma}.
It is also extended from probability measures to finite measures.

\begin{lemma}
  \label{lem:eGamma_self}
  %\lean{}
  %\leanok
  \uses{def:eGamma}
  For $\mu \in \mathcal M(\mathcal X)$, $E_\gamma(\mu, \mu) = 0$~.
\end{lemma}

\begin{proof}%\leanok
\uses{lem:statInfo_self}
Use Lemma~\ref{lem:statInfo_self}.
\end{proof}

\begin{lemma}
  \label{lem:eGamma_nonneg}
  %\lean{}
  %\leanok
  \uses{def:eGamma}
  For $\mu, \nu \in \mathcal M(\mathcal X)$, $E_\gamma(\mu, \nu) \ge 0$~.
\end{lemma}

\begin{proof}%\leanok
\uses{lem:statInfo_nonneg}
Use Lemma~\ref{lem:statInfo_nonneg}.
\end{proof}

\begin{theorem}[Data-processing inequality]
  \label{thm:data_proc_eGamma}
  \lean{ProbabilityTheory.eGamma_comp_le}
  \leanok
  \uses{def:eGamma}
  For $\mu, \nu \in \mathcal M(\mathcal X)$ and $\kappa : \mathcal X \rightsquigarrow \mathcal Y$ a Markov kernel, $E_\gamma(\kappa \circ \mu, \kappa \circ \nu) \le E_\gamma(\mu, \nu)$~.
\end{theorem}

\begin{proof}\leanok
\uses{thm:data_proc_statInfo}
Apply Theorem~\ref{thm:data_proc_statInfo}.
\end{proof}

\begin{lemma}[Integral form of the hockey-stick divergence]
  \label{lem:eGamma_eq_integral}
  %\lean{}
  %\leanok
  \uses{def:eGamma}
  For finite measures $\mu, \nu$ and $\gamma \in (0,+\infty)$,
  \begin{align*}
  E_\gamma(\mu, \nu)
  &= \nu\left[ x \mapsto \max \left\{0 , \frac{d \mu}{d\nu}(x) - \gamma \right\} \right] + \mu_{\perp \nu}(\mathcal X) & \text{ if } \mu(\mathcal X) \le \gamma \nu(\mathcal X)
  \: , \\
  E_\gamma(\mu, \nu)
  &= \nu\left[ x \mapsto \max \left\{0 , \gamma - \frac{d \mu}{d\nu}(x) \right\} \right] & \text{ if } \mu(\mathcal X) \ge \gamma \nu(\mathcal X)
  \: .
  \end{align*}
\end{lemma}

\begin{proof}%\leanok
\uses{thm:statInfo_eq_integral}
Apply Theorem~\ref{thm:statInfo_eq_integral}.
\end{proof}



\subsection{Total variation distance}

\begin{definition}
  \label{def:TV}
  %\lean{}
  %\leanok
  \uses{def:statInfo}
  The total variation distance between finite measures $\mu$ and $\nu$ is $\TV(\mu, \nu) = \mathcal I_{(1,1)}(\mu, \nu)$~.
\end{definition}

This is also equal to $2 I_{1/2}(\mu, \nu)$ and to $E_1(\mu, \nu)$.

\begin{lemma}
  \label{lem:tv_self}
  %\lean{}
  %\leanok
  \uses{def:TV}
  For $\mu \in \mathcal M(\mathcal X)$, $\TV(\mu, \mu) = 0$~.
\end{lemma}

\begin{proof}%\leanok
\uses{lem:statInfo_self}
Use Lemma~\ref{lem:statInfo_self}.
\end{proof}

\begin{lemma}
  \label{lem:tv_nonneg}
  %\lean{}
  %\leanok
  \uses{def:TV}
  For $\mu, \nu \in \mathcal M(\mathcal X)$, $\TV(\mu, \nu) \ge 0$~.
\end{lemma}

\begin{proof}%\leanok
\uses{lem:statInfo_nonneg}
Use Lemma~\ref{lem:statInfo_nonneg}.
\end{proof}


\begin{lemma}
  \label{lem:tv_le}
  %\lean{}
  %\leanok
  \uses{def:TV}
  For $\mu, \nu \in \mathcal M(\mathcal X)$, $\TV(\mu, \nu) \le \min\{\mu(\mathcal X), \nu(\mathcal X)\}$~.
\end{lemma}

\begin{proof}%\leanok
\uses{lem:statInfo_le}
Use Lemma~\ref{lem:statInfo_le}.
\end{proof}


\begin{theorem}[Data-processing inequality]
  \label{thm:tv_data_proc}
  %\lean{ProbabilityTheory.}
  %\leanok
  \uses{def:TV}
  For $\mu, \nu \in \mathcal M(\mathcal X)$ and $\kappa : \mathcal X \rightsquigarrow \mathcal Y$ a Markov kernel, $\TV(\kappa \circ \mu, \kappa \circ \nu) \le \TV(\mu, \nu)$~.
\end{theorem}

\begin{proof}%\leanok
\uses{thm:data_proc_statInfo}
Apply Theorem~\ref{thm:data_proc_statInfo}.
\end{proof}


\begin{lemma}
  \label{lem:tv_eq_sub_min}
  %\lean{}
  %\leanok
  \uses{def:TV}
  For finite measures $\mu, \nu$, for any measure $\zeta$ with $\mu \ll \zeta$ and $\nu \ll \zeta$~,
  \begin{align*}
  \TV(\mu, \nu)
  &= \min\{\mu(\mathcal X), \nu(\mathcal X)\} - \zeta\left[x \mapsto \min \left\{\frac{d \mu}{d\zeta}(x), \frac{d \nu}{d\zeta}(x)\right\}\right]
  \: .
  \end{align*}
  This holds in particular for $\zeta = P \circ \xi$.
\end{lemma}

\begin{proof}%\leanok
\uses{lem:statInfo_eq_sub_min}
Apply Lemma~\ref{lem:statInfo_eq_sub_min}.
\end{proof}


\begin{lemma}
  \label{lem:tv_eq_integral_abs_sub}
  %\lean{}
  %\leanok
  \uses{def:TV}
  For finite measures $\mu, \nu$, for any measure $\zeta$ with $\mu \ll \zeta$ and $\nu \ll \zeta$~,
  \begin{align*}
  \TV(\mu, \nu)
  &= - \frac{1}{2} \left\vert \mu(\mathcal X) - \nu(\mathcal X) \right\vert + 
  \frac{1}{2} \zeta\left[x \mapsto \left\vert \frac{d \mu}{d\zeta}(x) - \frac{d \nu}{d\zeta}(x)\right\vert\right] 
  \: .
  \end{align*}
  This holds in particular for $\zeta = P \circ \xi$.
\end{lemma}

\begin{proof}%\leanok
\uses{lem:statInfo_eq_integral_abs_sub}
Apply Lemma~\ref{lem:statInfo_eq_integral_abs_sub}.
\end{proof}


\begin{lemma}[Integral form of the total variation distance]
  \label{lem:tv_eq_integral}
  %\lean{}
  %\leanok
  \uses{def:TV}
  For finite measures $\mu, \nu$,
  \begin{align*}
  \TV(\mu, \nu)
  &= \nu\left[ x \mapsto \max \left\{0 , \frac{d \mu}{d\nu}(x) - 1 \right\} \right] + \mu_{\perp \nu}(\mathcal X) & \text{ if } \mu(\mathcal X) \le \nu(\mathcal X)
  \: , \\
  \TV(\mu, \nu)
  &= \nu\left[ x \mapsto \max \left\{0 , 1 - \frac{d \mu}{d\nu}(x) \right\} \right] & \text{ if } \mu(\mathcal X) \ge \nu(\mathcal X)
  \: .
  \end{align*}
\end{lemma}

\begin{proof}%\leanok
\uses{thm:statInfo_eq_integral}
Apply Theorem~\ref{thm:statInfo_eq_integral}.
\end{proof}


\begin{lemma}
  \label{lem:tv_eq_integral_abs}
  %\lean{}
  %\leanok
  \uses{def:TV}
  For finite measures $\mu, \nu$,
  \begin{align*}
  \TV(\mu, \nu)
  &= -\frac{1}{2} \left\vert \mu(\mathcal X) - \nu(\mathcal X)\right\vert + \frac{1}{2}\left( \nu\left[ x \mapsto \left\vert \frac{d \mu}{d\nu}(x) - 1 \right\vert \right]  + \mu_{\perp \nu}(\mathcal X)\right)
  \: .
  \end{align*}
\end{lemma}

\begin{proof}%\leanok
\uses{cor:statInfo_eq_integral_abs}
Apply Corollary~\ref{cor:statInfo_eq_integral_abs}.
\end{proof}


\begin{lemma}
  \label{lem:tv_eq_sub_inf_event}
  %\lean{}
  %\leanok
  \uses{def:TV}
  For finite measures $\mu, \nu$,
  \begin{align*}
  \TV(\mu, \nu)
  &= \min\{\mu(\mathcal X), \nu(\mathcal X)\} - \inf_{E \text{ event}} \left( \mu(E) + \nu(E^c) \right)
  \: .
  \end{align*}
\end{lemma}

\begin{proof}%\leanok
\uses{lem:statInfo_eq_sub_inf_event}
Apply Lemma~\ref{lem:statInfo_eq_sub_inf_event}.
\end{proof}


\begin{theorem}
  \label{thm:tv_eq_sup_sub_measure}
  %\lean{}
  %\leanok
  \uses{}
  Let $\mu, \nu$ be two finite measures on $\mathcal X$ with $\mu(\mathcal X) \leq \nu(\mathcal X)$.\\
  Then $TV(\mu, \nu) = \sup_{E \text{ event}} \left( \mu(E) - \nu(E) \right)$.
\end{theorem}

\begin{proof}%\leanok
\uses{lem:tv_eq_sub_inf_event}
Starting from Lemma~\ref{lem:tv_eq_sub_inf_event},
\begin{align*}
\TV(\mu, \nu)
&= \min\{\mu(\mathcal X), \nu(\mathcal X)\} - \inf_{E \text{ event}} \left( \mu(E) + \nu(E^c) \right)
\\
&= \mu(\mathcal X) - \inf_{E \text{ event}} \left( \mu(E^c) + \nu(E) \right)
\\
&= \mu(\mathcal X) - \mu(\mathcal X) - \inf_{E \text{ event}} \left( -\mu(E) + \nu(E) \right)
\\
&= \sup_{E \text{ event}} \left( \mu(E) - \nu(E) \right)
\: .
\end{align*}
\end{proof}


\begin{lemma}
  \label{lem:tv_eq_sup_aux}
  %\lean{}
  %\leanok
  \uses{def:TV}
  Let $\mathcal F = \{f : \mathcal X \to \mathbb{R} \mid \Vert f \Vert_\infty \le 1\}$.
  Then for $\mu, \nu$ finite measures with $\mu(\mathcal X) = \nu(\mathcal X)$, $\frac{1}{2} \sup_{f \in \mathcal F} \left( \mu[f] - \nu[f] \right) \le \TV(\mu, \nu)$.
\end{lemma}

\begin{proof}
\uses{lem:tv_eq_integral_abs_sub}
Let $p,q$ be the respective densities of $\mu, \nu$ with respect to $\xi=\mu+\nu$.
For any $f \in \mathcal F$,
\begin{align*}
\mu[f] - \nu[f]
= \int_x f(x)(p(x) - q(x)) \partial\xi
&\le \int_x \Vert f(x) \Vert_\infty \vert p(x) - q(x) \vert \partial\xi
\\
&\le \int_x \vert p(x) - q(x) \vert \partial\xi
= 2 \TV(\mu, \nu)
\: .
\end{align*}
The last equality is Lemma~\ref{lem:tv_eq_integral_abs_sub}.
\end{proof}


\begin{theorem}
  \label{thm:tv_eq_sup_sub_integral}
  %\lean{}
  %\leanok
  \uses{def:TV}
  Let $\mathcal F = \{f : \mathcal X \to \mathbb{R} \mid \Vert f \Vert_\infty \le 1\}$.
  Then for $\mu, \nu$ finite measures with $\mu(\mathcal X) = \nu(\mathcal X)$, $\TV(\mu, \nu) = \frac{1}{2} \sup_{f \in \mathcal F} \left( \mu[f] - \nu[f] \right)$.
\end{theorem}

\begin{proof}
\uses{lem:tv_eq_sup_aux, thm:tv_eq_sup_sub_measure}
Lemma~\ref{lem:tv_eq_sup_aux} gives $\TV(\mu, \nu) \ge \frac{1}{2} \sup_{f \in \mathcal F} \left( \mu[f] - \nu[f] \right)$.

The bayes estimator is a function of the form $x \mapsto \mathbb{I}\{x \in E\}$ for some event $E$, hence it belongs to $\mathcal F$. This gives equality.
\end{proof}


\begin{lemma}
  \label{lem:tv_data_proc_event}
  %\lean{}
  %\leanok
  \uses{def:TV}
  Let $\mu, \nu$ be two measures on $\mathcal X$ and let $E$ be an event. Let $\mu_E$ and $\nu_E$ be the two Bernoulli distributions with respective means $\mu(E)$ and $\nu(E)$.
  Then $\TV(\mu, \nu) \ge \TV(\mu_E, \nu_E)$.
\end{lemma}

\begin{proof}
\uses{cor:statInfo_data_proc_event}
Apply Corollary~\ref{cor:statInfo_data_proc_event}.
\end{proof}
