\section{Sequential testing}

For $\mu \in \mathcal M(\mathcal X)$, let $\mu^{\otimes \mathbb{N}}$ be the corresponding product measure on $\mathcal X^{\mathbb{N}}$.
For $n \in \mathbb{N}$, let $\mathcal F_n$ be the $\sigma$-algebra of $\mathcal X^{\mathbb{N}}$ generated by the projection to the first $n$ coordinates. Denote that projection by $\pi_{[n]}$.
$\mathcal F = (\mathcal F_n)_{n \in \mathbb{N}}$ is a filtration.
Let $\tau$ be a stopping time with respect to $\mathcal F$. $\mu_\tau$ denotes the restriction of the measure $\mu^{\otimes \mathbb{N}}$ to $\mathcal F_\tau$, the sigma-algebra generated by the stopping time $\tau$.

In sequential testing, two components need to be designed for estimation: the stopping time $\tau$ and the $\mathcal F_\tau$-measurable estimator that returns an answer given the data available at $\tau$.
We will be interested in two performance metrics:
\begin{itemize}
  \item The risk, where the data generating kernel is $P_\tau$ which takes values $\mu_\tau$ and $\nu_\tau$.
  \item The stopping time $\tau$ (through its mean, other moments, or high probability bounds).
\end{itemize}
The idea is that a good estimator stops quickly with low risk.

\begin{lemma}
  \label{lem:llr_filtration_nat}
  %\lean{}
  %\leanok
  %\uses{}
  For $\mu, \nu \in \mathcal P(\mathcal X)$ with $\mu \ll \nu$ and $n \in \mathbb{N}$, $\nu^{\otimes \mathbb{N}}_{| \mathcal F_n}$-almost surely,
  $\frac{d \mu^{\otimes \mathbb{N}}_{| \mathcal F_n}}{d \nu^{\otimes \mathbb{N}}_{| \mathcal F_n}}(x) = \prod_{m=1}^n \frac{d \mu}{d \nu}(x_m)$.
\end{lemma}

\begin{proof}%\leanok
\uses{lem:rnDeriv_map_eq_rnDeriv_trim}
By Lemma~\ref{lem:rnDeriv_map_eq_rnDeriv_trim}, $\frac{d \mu^{\otimes \mathbb{N}}_{| \mathcal F_n}}{d \nu^{\otimes \mathbb{N}}_{| \mathcal F_n}}(x) = \frac{d \pi_{[n]*}\mu^{\otimes \mathbb{N}}}{d \pi_{[n]*}\nu^{\otimes \mathbb{N}}}(\pi_{[n]}(x))$ a.s. . The pushforwards simplify: $\pi_{[n]*}\nu^{\otimes \mathbb{N}} = \nu^{\otimes n}$.
\end{proof}

\begin{lemma}
  \label{lem:llr_stopping_time}
  %\lean{}
  %\leanok
  %\uses{}
  For $\mu, \nu \in \mathcal P(\mathcal X)$ with $\mu \ll \nu$, $\nu_\tau$-almost surely,
  $\frac{d \mu_\tau}{d \nu_\tau}(x) = \prod_{n=1}^\tau \frac{d \mu}{d \nu}(x_n)$.
\end{lemma}

\begin{proof}
\uses{lem:rnDeriv_trim_of_ac, lem:llr_filtration_nat}
It suffices to show the equality of their integrals on $\mathcal F_\tau$-measurable sets. Let $E$ be such a set.
Then $E = \sum_{n=1}^\infty E \cap \{\tau = n\}$~. For $n \in \mathbb{N}$, let $E_n = E \cap \{\tau = n\}$.

We show that $\nu_\tau \left[\mathbb{I}(E_n) \frac{d \mu_\tau}{d \nu_\tau}\right] = \nu_\tau\left[x \mapsto \mathbb{I}(E_n)\prod_{m=1}^n \frac{d \mu}{d \nu}(x_m)\right]$.
The result for $E$ then follows from the monotone convergence theorem.
The product $\prod_{m=1}^n \frac{d \mu}{d \nu}(x_m)$ is equal to $\frac{d \mu^{\otimes \mathbb{N}}_{| \mathcal F_n}}{d \nu^{\otimes \mathbb{N}}_{| \mathcal F_n}}$ by Lemma~\ref{lem:llr_filtration_nat}.

$E_n$ is $\mathcal F_n$-measurable, hence (using Lemma~\ref{lem:rnDeriv_trim_of_ac})
TODO

\end{proof}

\begin{theorem}
  \label{thm:kl_stopping_time}
  %\lean{}
  %\leanok
  \uses{def:KL}
  For $\mu, \nu \in \mathcal P(\mathcal X)$, $\KL(\mu_\tau, \nu_\tau) = \mu[\tau] \KL(\mu, \nu)$.
\end{theorem}

\begin{proof}
\uses{lem:llr_stopping_time}
TODO: need Wald's first identity.
\end{proof}

\begin{lemma}
  \label{lem:renyiMeasure_stopping_time}
  %\lean{}
  %\leanok
  \uses{def:renyiMeasure}
  For $\mu, \nu$ two probability measures on $\mathcal X$ and $\alpha \in (0,1)$, $(\mu^{(\alpha, \nu)})_\tau = (\mu_\tau)^{(\alpha, \nu_\tau)}$.
\end{lemma}

\begin{proof}
This is a guess, I did not check it. TODO
\end{proof}

\begin{theorem}
  \label{thm:renyi_stopping_time}
  %\lean{}
  %\leanok
  \uses{def:KL, def:Renyi, def:renyiMeasure}
  For $\mu, \nu$ two probability measures on $\mathcal X$ and $\alpha \in (0,1)$,
  $R_\alpha(\mu_\tau, \nu_\tau) = \mu^{\alpha, \nu}[\tau] R_\alpha(\mu, \nu)$.
\end{theorem}

\begin{proof}
\uses{thm:kl_stopping_time, cor:renyi_eq_add_kl, lem:renyiMeasure_stopping_time}
From Corollary~\ref{cor:renyi_eq_add_kl}, $(1 - \alpha) R_\alpha(\mu, \nu) = \alpha \KL(\mu^{(\alpha, \nu)}, \mu) + (1 - \alpha)\KL(\mu^{(\alpha, \nu)}, \nu)$, hence we can use Theorem~\ref{thm:kl_stopping_time} to write
\begin{align*}
\mu^{\alpha, \nu}[\tau] R_\alpha(\mu, \nu)
&= \frac{1}{1 - \alpha}\left(\alpha \mu^{\alpha, \nu}[\tau]\KL(\mu^{(\alpha, \nu)}, \mu) + (1 - \alpha)\mu^{\alpha, \nu}[\tau]\KL(\mu^{(\alpha, \nu)}, \nu) \right)
\\
&= \frac{1}{1 - \alpha}\left(\alpha \KL((\mu^{(\alpha, \nu)})_\tau, \mu_\tau) + (1 - \alpha)\KL((\mu^{(\alpha, \nu)})_\tau, \nu_\tau) \right)
\end{align*}
It then suffices to remark that $(\mu^{(\alpha, \nu)})_\tau = (\mu_\tau)^{(\alpha, \nu_\tau)}$, thanks to Lemma~\ref{lem:renyiMeasure_stopping_time}.

TODO: if that lemma does not hold, we still have the inequality $R_\alpha(\mu_\tau, \nu_\tau) \le \mu^{\alpha, \nu}[\tau] R_\alpha(\mu, \nu)$ by Lemma~\ref{lem:renyi_eq_inf_add_kl}.
\end{proof}