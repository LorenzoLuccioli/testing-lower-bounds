\chapter{Testing error lower bounds}

In simple hypothesis testing, a sample from an unknown measure on $\mathcal X$ among $\{\mu, \nu\}$ is observed, and the goal of the test is to return the measure from which the sample came.
In typical scenarios, $\mathcal X$ is a product space $\mathcal Y^n$, $\mu$ and $\nu$ are product measures and the observation is a collection of $n$ i.i.d. samples.

A test is a measurable function $\phi : \mathcal X \to \{0,1\}$, which can equivalently be described by the even $E = \{\phi = 1\}$. If we decide that the test should return 0 if the sample came from $\mu$ and 1 if it came from $\nu$, the two error probabilities of the test are $\mu(E)$ and $\nu(E^c)$. A good test minimizes these (or a given combination of these).

In this section, we prove lower bounds on functions of $\mu(E)$ and $\nu(E^c)$, for any event $E$.
For the sum $\mu(E) + \nu(E^c)$, the value of the optimal test is exactly $1 - \TV(\mu, \nu)$, as shown in the following theorem.

\begin{theorem}
  \label{thm:testing_eq_tv}
  %\lean{}
  %\leanok
  \uses{def:TV}
  Let $\mu, \nu$ be two probability measures on $\mathcal X$. Then $\inf_{E \text{ event}}\{\mu(E) + \nu(E^c)\} = 1 - \TV(\mu, \nu)$~.
\end{theorem}

\begin{proof}
\uses{thm:tv_eq_sup_sub_measure}
By Theorem~\ref{thm:tv_eq_sup_sub_measure},
\begin{align*}
-TV(\mu, \nu)
= -\sup_{E \text{ event}} \left( \mu(E) - \nu(E) \right)
=\inf_{E \text{ event}} \left(\nu(E) - \mu(E) \right)
= \inf_{E \text{ event}} \left( \nu(E) + \mu(E^c) \right) - 1
\: .
\end{align*}
\end{proof}

The drawback of $\TV$ is that it is not easy to manipulate. In particular, it does not tensorize. In order to quantify error probabilities on product spaces like $\mu^{\otimes n}(E) + \nu^{\otimes n}(E^c)$, we use other divergences.

\section{Generic lower bounds}

\begin{lemma}
  \label{lem:testing_bound_renyi_mean}
  %\lean{}
  %\leanok
  \uses{def:Renyi}
  Let $\mu, \nu$ be two probability measures on $\mathcal X$ and $E$ an event. Let $\alpha \in (0,1)$. Then
  \begin{align*}
  \mu(E)^\alpha + \nu(E^c)^{1 - \alpha}
  \ge \exp\left(-(1 - \alpha) R_{\alpha}(\mu, \nu)\right)
  \: .
  \end{align*}
\end{lemma}

\begin{proof}
\uses{lem:renyi_data_proc_event}
Let $\mu_E$ and $\nu_E$ be the two Bernoulli distributions with respective means $\mu(E)$ and $\nu(E)$.
By Lemma~\ref{lem:renyi_data_proc_event}, $R_\alpha(\mu, \nu) \ge R_\alpha(\mu_E, \nu_E)$. That divergence is
\begin{align*}
R_\alpha(\mu_E, \nu_E)
&= \frac{1}{\alpha - 1}\log \left(\mu(E)^\alpha \nu(E)^{1 - \alpha}
  + \mu(E^c)^\alpha \nu(E^c)^{1 - \alpha}\right)
\\
&\ge \frac{1}{\alpha - 1}\log \left(\mu(E)^\alpha + \nu(E^c)^{1 - \alpha}\right)
\: .
\end{align*}
\end{proof}

\begin{corollary}
  \label{cor:testing_bound_hellinger}
  %\lean{}
  %\leanok
  \uses{def:Hellinger, def:Renyi}
  Let $\mu, \nu$ be two probability measures on $\mathcal X$ and $E$ an event. Then
  \begin{align*}
  \sqrt{\mu(E)} + \sqrt{\nu(E^c)}
  \ge \exp\left(-\frac{1}{2} R_{1/2}(\mu, \nu)\right)
  = 1 - \Hsq(\mu, \nu)
  \: .
  \end{align*}
\end{corollary}

\begin{proof}
\uses{lem:testing_bound_renyi_mean, lem:renyi_half_eq_log_hellinger}
The inequality is an application of Lemma~\ref{lem:testing_bound_renyi_mean} for $\alpha = 1/2$. The equality is Lemma~\ref{lem:renyi_half_eq_log_hellinger}.
\end{proof}

\section{Change of measure}

\begin{lemma}
  \label{lem:llr_change_measure}
  \lean{ProbabilityTheory.measure_sub_le_measure_mul_exp'}
  \leanok
  %\uses{}
  Let $\mu, \nu$ be two measures on $\mathcal X$ with $\mu \ll \nu$ and let $E$ be an event on $\mathcal X$. Let $\beta \in \mathbb{R}$. Then
  \begin{align*}
  \nu(E) e^{\beta} \ge \mu(E) - \mu\left\{ \log\frac{d \mu}{d \nu} > \beta \right\} \: .
  \end{align*}
\end{lemma}

\begin{proof}\leanok
\begin{align*}
\nu(E)
\ge \mu\left[\mathbb{I}(E) e^{- \log\frac{d \mu}{d \nu} }\right]
&\ge \mu\left[\mathbb{I}\left(E \cap \left\{\log\frac{d \mu}{d \nu} \le \beta\right\}\right) e^{- \log\frac{d \mu}{d \nu} }\right]
\\
&\ge e^{- \beta}\mu\left(E \cap \left\{\log\frac{d \mu}{d \nu} \le \beta\right\}\right)
\\
&\ge e^{- \beta}\left( \mu(E) - \mu\left\{ \log\frac{d \mu}{d \nu} > \beta \right\} \right)
\: .
\end{align*}
\end{proof}

\begin{corollary}
  \label{cor:kl_change_measure}
  %\lean{}
  %\leanok
  \uses{def:KL}
  Let $\mu, \nu$ be two measures on $\mathcal X$ and let $E$ be an event on $\mathcal X$. Let $\beta \in \mathbb{R}$. Then
  \begin{align*}
  \nu(E) e^{\KL(\mu, \nu) + \beta} \ge \mu(E) - \mu\left\{ \log\frac{d \mu}{d \nu} - \KL(\mu, \nu) > \beta \right\} \: .
  \end{align*}
\end{corollary}

\begin{proof}
\uses{lem:llr_change_measure}
Use Lemma~\ref{lem:llr_change_measure} with the choice $\KL(\mu, \nu) + \beta$ for $\beta$.
\end{proof}

\begin{lemma}
  \label{lem:llr_change_measure_variance}
  %\lean{}
  %\leanok
  \uses{}
  Let $\mu, \nu$ be two measures on $\mathcal X$ such that $\mu\left[\left(\log\frac{d \mu}{d \nu}\right)^2\right] < \infty$. Let $E$ be an event on $\mathcal X$ and let $\beta > 0$. Then
  \begin{align*}
  \nu(E) e^{\KL(\mu, \nu) + \sqrt{\Var_\mu[\log\frac{d \mu}{d \nu}]\beta}} \ge \mu(E) - \frac{1}{\beta} \: .
  \end{align*}
\end{lemma}

\begin{proof}%\leanok
\uses{lem:llr_change_measure}
Use Lemma~\ref{lem:llr_change_measure} with the choice $\KL(\mu, \nu) + \sqrt{\Var_\mu[\log\frac{d \mu}{d \nu}]\beta}$ for $\beta$ and bound the probability of deviation of the log-likelihood ratio with Chebychev's inequality.
\end{proof}

\begin{lemma}
  \label{lem:renyi_chernoff_bound}
  \lean{ProbabilityTheory.measure_llr_gt_renyiDiv_le_exp}
  \leanok
  \uses{def:Renyi}
  For $\mu, \nu$ finite measures and $\alpha, \beta > 0$,
  \begin{align*}
  \mu\left\{ \log\frac{d \mu}{d \nu} > R_{1+\alpha}(\mu, \nu) + \beta \right\}
  \le e^{- \alpha \beta}
  \: .
  \end{align*}
\end{lemma}

\begin{proof}\leanok
\uses{lem:renyi_cgf_2}
This is a Chernoff bound, using that the cumulant generating function of $\log\frac{d\mu}{d\nu}$ under $\mu$ has value $\alpha R_{1+\alpha}(\mu, \nu)$ at $\alpha$ by Lemma~\ref{lem:renyi_cgf_2}.
\begin{align*}
\mu\left\{ \log\frac{d \mu}{d \nu} > R_{1+\alpha}(\mu, \nu) + \beta \right\}
&= \mu\left\{ \exp\left(\alpha\log\frac{d \mu}{d \nu}\right) > \exp\left(\alpha R_{1+\alpha}(\mu, \nu) + \alpha \beta\right) \right\}
\\
&\le e^{-\alpha R_{1+\alpha}(\mu, \nu) - \alpha \beta} \mu\left[\left(\frac{d \mu}{d \nu}\right)^\alpha \right]
\end{align*}
Then $\mu\left[\left(\frac{d \mu}{d \nu}\right)^\alpha \right] = \nu\left[\left(\frac{d \mu}{d \nu}\right)^{1+\alpha} \right] = e^{\alpha R_{1+\alpha}(\mu, \nu)}$.
\end{proof}

\begin{lemma}
  \label{lem:renyi_change_measure}
  \lean{ProbabilityTheory.measure_sub_le_measure_mul_exp_renyiDiv}
  \leanok
  \uses{def:Renyi}
  Let $\mu, \nu$ be two finite measures on $\mathcal X$ and let $E$ be an event on $\mathcal X$. Let $\alpha,\beta > 0$. Then
  \begin{align*}
  \nu(E) e^{R_{1+\alpha}(\mu, \nu) + \beta} \ge \mu(E) - e^{-\alpha \beta} \: .
  \end{align*}
\end{lemma}

\begin{proof}\leanok
\uses{lem:llr_change_measure, lem:renyi_chernoff_bound}
Use Lemma~\ref{lem:llr_change_measure} with the choice $R_{1+\alpha}(\mu, \nu) + \beta$ for $\beta$. Then apply Lemma~\ref{lem:renyi_chernoff_bound}.
\end{proof}

\begin{lemma}
  \label{lem:llr_change_measure_add}
  \lean{ProbabilityTheory.one_sub_le_add_measure_mul_exp}
  \leanok
  %\uses{}
  Let $\mu, \nu, \xi$ be three probability measures on $\mathcal X$ and let $E$ be an event on $\mathcal X$. Let $\beta_1, \beta_2 \in \mathbb{R}$. Then
  \begin{align*}
  \mu(E) e^{\beta_1} + \nu(E^c) e^{\beta_2} \ge 1 - \xi\left\{ \log\frac{d \xi}{d \mu} > \beta_1 \right\} - \xi\left\{ \log\frac{d \xi}{d \nu} > \beta_2 \right\} \: .
  \end{align*}
\end{lemma}

\begin{proof}\leanok
\uses{lem:llr_change_measure}
Two applications of Lemma~\ref{lem:llr_change_measure}, then sum them and use $\xi(E)+\xi(E^c) = 1$.
\end{proof}

\begin{lemma}
  \label{lem:change_measure_variance_add}
  \lean{ProbabilityTheory.one_sub_exp_le_add_measure_mul_exp_max_renyiDiv} %TODO: this seems like a mistake, the lemma doesn't correspond to this statement. Moreover, the referenced lean lemma is already linked to the statement below and it's more similar to that, even if not exactly the same.
  \leanok
  \uses{}
  Let $\mu, \nu, \xi$ be three probability measures on $\mathcal X$ and let $E$ be an event on $\mathcal X$. For $\beta > 0$~,
  \begin{align*}
  \mu(E) e^{\KL(\xi, \mu) + \sqrt{\beta \Var_{\xi}\left[\log\frac{d\xi}{d\mu}\right]}} + \nu(E^c) e^{\KL(\xi, \nu) + \sqrt{\beta \Var_{\xi}\left[\log\frac{d\xi}{d\nu}\right]}}
  \ge 1 - \frac{2}{\beta} \: .
  \end{align*}
\end{lemma}

\begin{proof}\leanok
\uses{lem:llr_change_measure_add}
Use Lemma~\ref{lem:llr_change_measure_add} with the choices $\KL(\xi, \mu) + \sqrt{\beta \Var_{\xi}\left[\log\frac{d\xi}{d\mu}\right]}$ and $\KL(\xi, \nu) + \sqrt{\beta \Var_{\xi}\left[\log\frac{d\xi}{d\nu}\right]}$ for $\beta_1$ and $\beta_2$.
Then use Chebyshev's inequality to bound the probabilities of deviation of the log-likelihood ratios.
\end{proof}

\begin{lemma}
  \label{lem:renyi_change_measure_add}
  \lean{ProbabilityTheory.one_sub_exp_le_add_measure_mul_exp_max_renyiDiv}
  \leanok
  \uses{def:Renyi}
  Let $\mu, \nu, \xi$ be three probability measures on $\mathcal X$ and let $E$ be an event on $\mathcal X$. Let $\alpha, \beta \ge 0$. Then
  \begin{align*}
  \mu(E) e^{R_{1+\alpha}(\xi, \mu) + \beta} + \nu(E^c) e^{R_{1+\alpha}(\xi, \nu) + \beta} \ge 1 - 2 e^{-\alpha \beta} \: .
  \end{align*}
\end{lemma}

\begin{proof}\leanok
\uses{lem:llr_change_measure_add, lem:renyi_chernoff_bound}
Use Lemma~\ref{lem:llr_change_measure_add} with the choices $R_{1+\alpha}(\xi, \mu) + \beta$ and $R_{1+\alpha}(\xi, \nu) + \beta$ for $\beta_1$ and $\beta_2$.
Then apply Lemma~\ref{lem:renyi_chernoff_bound}.
\end{proof}


\section{Lower bounds on the maximum}

\begin{lemma}
  \label{lem:testing_bound_tv_hellinger_max}
  %\lean{}
  %\leanok
  \uses{def:TV, def:Hellinger}
  Let $\mu, \nu$ be two probability measures on $\mathcal X$ and $E$ an event. Then
  \begin{align*}
  \max\{\mu(E), \nu(E^c)\}
  \ge \frac{1}{2}(1 - \TV(\mu, \nu))
  \ge \frac{1}{4}(1 - \Hsq(\mu, \nu))^2
  \: .
  \end{align*}
\end{lemma}

\begin{proof}
\uses{thm:testing_eq_tv, cor:one_sub_hellinger_squared_le_one_sub_tv}
The first inequality is Theorem~\ref{thm:testing_eq_tv} with $\mu(E) + \nu(E^c) \le 2 \max\{\mu(E), \nu(E^c)\}$.
The second inequality is a consequence of Corollary~\ref{cor:one_sub_hellinger_squared_le_one_sub_tv}.
\end{proof}

\begin{lemma}
  \label{lem:testing_bound_renyi_max}
  %\lean{}
  %\leanok
  \uses{def:Renyi}
  Let $\mu, \nu$ be two probability measures on $\mathcal X$ and $E$ an event. Let $\alpha \in (0,1)$. Then
  \begin{align*}
  \min\{\alpha, 1 - \alpha\} \log\frac{1}{\max\{\mu(E), \nu(E^c)\}} \le (1 - \alpha) R_{\alpha}(\mu, \nu)  + \log 2 \: .
  \end{align*}
\end{lemma}

\begin{proof}
\uses{lem:testing_bound_renyi_mean}
Use Lemma~\ref{lem:testing_bound_renyi_mean} and remark that $\mu(E)^\alpha + \nu(E^c)^{1 - \alpha} \le 2\max\{\mu(E), \nu(E^c)\}^{\min\{\alpha, 1 - \alpha\}}$.
\end{proof}


\begin{lemma}
  \label{lem:testing_bound_renyi_one_add}
  \lean{ProbabilityTheory.exp_neg_chernoffDiv_le_add_measure}
  \leanok
  \uses{def:Renyi, def:Chernoff}
  Let $\mu, \nu$ be two probability measures on $\mathcal X$ and let $E$ be an event on $\mathcal X$. Let $\alpha > 0$. Then
  \begin{align*}
  \mu(E) + \nu(E^c) \ge \frac{1}{2}\exp\left( - C_{1+\alpha}(\mu, \nu) - \frac{\log 4}{\alpha}\right) \: .
  \end{align*}
\end{lemma}

\begin{proof}%\leanok
\uses{lem:renyi_change_measure_add}
\end{proof}


\section{Product spaces}

\begin{corollary}
  \label{cor:testing_bound_renyi_n}
  %\lean{}
  %\leanok
  \uses{def:Renyi}
  Let $\mu, \nu$ be two probability measures on $\mathcal X$ and $E$ an event of $\mathcal X^n$. Let $\alpha \in (0,1)$. Then
  \begin{align*}
  \min\{\alpha, 1 - \alpha\} \log\frac{1}{\max\{\mu^{\otimes n}(E), \nu^{\otimes n}(E^c)\}} \le (1 - \alpha) n R_{\alpha}(\mu, \nu)  + \log 2 \: .
  \end{align*}
\end{corollary}

\begin{proof}
\uses{lem:testing_bound_renyi_max, lem:renyi_prod_n}
Use Lemma~\ref{lem:renyi_prod_n} in Lemma~\ref{lem:testing_bound_renyi_max}.
\end{proof}

\begin{lemma}
  \label{lem:testing_bound_renyi_one_add_n}
  %\lean{}
  %\leanok
  \uses{def:Renyi, def:Chernoff}
  Let $\mu, \nu$ be two probability measures on $\mathcal X$, let $n \in \mathbb{N}$ and let $E$ be an event on $\mathcal X^n$. For all $\alpha > 0$,
  \begin{align*}
  \mu^{\otimes n}(E) + \nu^{\otimes n}(E^c) \ge \frac{1}{2}\exp\left( - n C_{1+\frac{\alpha}{\sqrt{n}}}(\mu, \nu) - \frac{\log 4}{\alpha}\sqrt{n}\right) \: .
  \end{align*}
\end{lemma}

\begin{proof}
\uses{lem:renyi_prod_n, lem:testing_bound_renyi_one_add}
Use Lemma~\ref{lem:renyi_prod_n} in Lemma~\ref{lem:testing_bound_renyi_one_add}. TODO: add a lemma for tensorization of the Chernoff divergence.
\end{proof}

\begin{theorem}
  \label{thm:testing_bound_chernoff}
  %\lean{}
  %\leanok
  \uses{def:Chernoff}
  Let $\mu, \nu$ be two probability measures on $\mathcal X$ and let $(E_n)_{n \in \mathbb{N}}$ be events on $\mathcal X^n$. For all $\gamma \in (0,1)$,
  \begin{align*}
  \limsup_{n \to +\infty} \frac{1}{n}\log \frac{1}{\gamma \mu^{\otimes n}(E_n) + (1 - \gamma)\nu^{\otimes n}(E_n^c)}
  \le C_1(\mu, \nu)
  \: .
  \end{align*}
\end{theorem}

\begin{proof}
\uses{cor:kl_change_measure, thm:kl_pi}
Let $\xi$ be a probability measure on $\mathcal X$ and $\beta > 0$. By Corollary~\ref{cor:kl_change_measure},
\begin{align*}
\mu^{\otimes n}(E_n) e^{\KL(\xi^{\otimes n}, \mu^{\otimes n}) + n\beta}
&\ge \xi^{\otimes n}(E_n) - \xi^{\otimes n}\left\{ \log\frac{d \xi^{\otimes n}}{d \mu^{\otimes n}} - \KL(\xi^{\otimes n}, \mu^{\otimes n}) > n\beta \right\}
\: , \\
\nu^{\otimes n}(E_n^c) e^{\KL(\xi^{\otimes n}, \nu^{\otimes n}) + n\beta}
&\ge \xi^{\otimes n}(E_n^c) - \xi^{\otimes n}\left\{ \log\frac{d \xi^{\otimes n}}{d \nu^{\otimes n}} - \KL(\xi^{\otimes n}, \nu^{\otimes n}) > n\beta \right\}
\: .
\end{align*}
We sum both inequalities with weights $\gamma$ and $1-\gamma$ respectively and use that each $\KL$ on the left is less than their max, as well as $\xi^{\otimes n}(E_n) + \xi^{\otimes n}(E_n^c) = 1$.
\begin{align*}
&e^{n\beta} (\gamma\mu^{\otimes n}(E_n) + (1-\gamma)\nu^{\otimes n}(E_n^c)) e^{\max\{\KL(\xi^{\otimes n}, \mu^{\otimes n}), \KL(\xi^{\otimes n}, \nu^{\otimes n})\}}
\\
&\ge \min\{\gamma, 1-\gamma\} - \gamma\xi^{\otimes n}\left\{ \log\frac{d \xi^{\otimes n}}{d \mu^{\otimes n}} - \KL(\xi^{\otimes n}, \mu^{\otimes n}) > n\beta \right\}
  - (1 - \gamma)\xi^{\otimes n}\left\{ \log\frac{d \xi^{\otimes n}}{d \nu^{\otimes n}} - \KL(\xi^{\otimes n}, \nu^{\otimes n}) > n\beta \right\}
\: .
\end{align*}
Let $p_{n,\mu}(\beta)$ and $p_{n, \nu}(\beta)$ be the two probabilities on the right hand side. By the law of large numbers, both tend to 0 when $n$ tends to $+\infty$.
In particular, for $n$ large enough, the right hand side is positive and we can take logarithms on both sides. We also use the tensorization of $\KL$ (Theorem~\ref{thm:kl_pi}).
\begin{align*}
& n \max\{\KL(\xi, \mu), \KL(\xi, \nu)\}
\\
&\ge \log \frac{1}{\gamma \mu^{\otimes n}(E_n) + (1 - \gamma)\nu^{\otimes n}(E_n^c)} + \log (\min\{\gamma, 1-\gamma\} - \gamma p_{n, \mu}(\beta) - (1 - \gamma) p_{n, \nu}(\beta)) - n\beta
\end{align*}
For $n \to +\infty$,
\begin{align*}
\max\{\KL(\xi, \mu), \KL(\xi, \nu)\}
\ge \limsup_{n \to + \infty}\frac{1}{n}\log \frac{1}{\gamma \mu^{\otimes n}(E_n) + (1 - \gamma)\nu^{\otimes n}(E_n^c)} - \beta
\end{align*}
Since $\beta > 0$ is arbitrary, we can take a supremum over $\beta$ on the right.
\end{proof}