\section{Hellinger alpha-divergences}

\begin{definition}[Hellinger $\alpha$-divergence]
  \label{def:hellingerAlpha}
  \lean{ProbabilityTheory.hellingerDiv}
  \leanok
  \uses{def:KL, def:fDiv, lem:fDiv_properties, lem:kl_properties}
  Let $\mu, \nu$ be two measures on $\mathcal X$. The Hellinger divergence of order $\alpha \in [0,+\infty)$ between $\mu$ and $\nu$ is
  \begin{align*}
  \He_\alpha(\mu, \nu) = \left\{
  \begin{array}{ll}
    \nu \{x \mid \frac{d\mu}{d\nu} (x) = 0\} & \text{for } \alpha = 0 \\
    \KL(\mu, \nu) & \text{for } \alpha = 1
    \\
    D_{f_\alpha}(\mu, \nu) & \text{for } \alpha \in (0,+\infty) \backslash \{1\}
  \end{array}\right.
  \end{align*}
  with $f_\alpha : x \mapsto \frac{x^{\alpha} - 1}{\alpha - 1}$.
\end{definition}


\begin{lemma}
  \label{lem:hellingerAlpha_ne_top_of_lt_one}
  \lean{ProbabilityTheory.hellingerDiv_ne_top_of_lt_one}
  \leanok
  \uses{def:hellingerAlpha}
  For $\alpha \in [0, 1)$ and finite measures $\mu, \nu$, $\He_\alpha(\mu, \nu) < \infty$.
\end{lemma}

\begin{proof}\leanok
\uses{}
\end{proof}


\begin{lemma}
  \label{lem:hellingerAlpha_eq_integral}
  \lean{ProbabilityTheory.hellingerDiv_eq_integral_of_ne_top''}
  \leanok
  \uses{def:hellingerAlpha}
  For $\alpha \in (0,1)\cup(1, \infty)$, $\mu$ a finite measure and $\nu$ a probability measure, if $\He_\alpha(\mu, \nu) < \infty$ then
  \begin{align*}
  \He_\alpha(\mu, \nu) = \frac{1}{\alpha - 1} \left( \int_x \left(\frac{d \mu}{d \nu}(x)\right)^\alpha \partial \nu - 1 \right)
  \: .
  \end{align*}
\end{lemma}

\begin{proof}\leanok
\uses{}
\end{proof}


\begin{lemma}
  \label{lem:integral_rpow_rnDeriv}
  \lean{ProbabilityTheory.integral_rpow_rnDeriv}
  \leanok
  \uses{}
  For $\alpha \in (0,1)\cup(1, \infty)$, $\mu, \nu$ two sigma-finite measures,
  \begin{align*}
  \int_x \left(\frac{d \mu}{d \nu}(x)\right)^\alpha \partial \nu
  = \int_x \left(\frac{d \nu}{d \mu}(x)\right)^{1 - \alpha} \partial \mu
  \: .
  \end{align*}
\end{lemma}

\begin{proof}\leanok
\uses{}
\begin{align*}
  \int_x \left(\frac{d \mu}{d \nu}(x)\right)^\alpha \partial \nu
  &= \int_x \left(\frac{d \mu}{d (\mu + \nu)}(x) \left(\frac{d \nu}{d (\mu + \nu)}(x)\right)^{-1} \right)^\alpha \partial \nu \\
  &= \int_x \left(\frac{d \mu}{d (\mu + \nu)}(x) \left(\frac{d \nu}{d (\mu + \nu)}(x)\right)^{-1} \right)^\alpha \frac{d \nu}{d (\mu + \nu)}(x) \partial (\mu + \nu) \\
  &= \int_x \left(\frac{d \mu}{d (\mu + \nu)}(x) \left(\frac{d \nu}{d (\mu + \nu)}(x)\right)^{-1} \right)^{\alpha - 1} \frac{d \mu}{d (\mu + \nu)}(x) \partial (\mu + \nu) \\
  &= \int_x \left(\frac{d \mu}{d (\mu + \nu)}(x) \left(\frac{d \nu}{d (\mu + \nu)}(x)\right)^{-1} \right)^{\alpha - 1} \partial \mu \\
  &= \int_x \left(\frac{d \nu}{d \mu}(x)\right)^{1 - \alpha} \partial \mu
  \: .
  \end{align*}
\end{proof}


\begin{lemma}
  \label{lem:hellingerAlpha_symm}
  \lean{ProbabilityTheory.hellingerDiv_symm'}
  \leanok
  \uses{def:hellingerAlpha}
  For $\alpha \in (0, 1)$ and finite measures $\mu, \nu$ with $\mu(\mathcal X) = \nu(\mathcal X)$, $(1 - \alpha) \He_\alpha(\mu, \nu) = \alpha \He_{1 - \alpha}(\nu, \mu)$.
\end{lemma}

\begin{proof}\leanok
\uses{lem:integral_rpow_rnDeriv}
Use Lemma~\ref{lem:integral_rpow_rnDeriv}.
\end{proof}


\begin{definition}[Conditional Hellinger $\alpha$-divergence]
  \label{def:condHellingerAlpha}
  \lean{ProbabilityTheory.condHellingerDiv}
  \leanok
  \uses{def:condFDiv, lem:fDiv_properties}
  Let $\mu$ be a measure on $\mathcal X$ and $\kappa, \eta : \mathcal X \rightsquigarrow \mathcal Y$ be two kernels. The conditional Hellinger divergence of order $\alpha \in (0,+\infty) \backslash \{1\}$ between $\kappa$ and $\eta$ conditionally to $\mu$ is
  \begin{align*}
  \He_\alpha(\kappa, \eta \mid \mu) = D_{f_\alpha}(\kappa, \eta \mid \mu) \: ,
  \end{align*}
  for $f_\alpha : x \mapsto \frac{x^{\alpha} - 1}{\alpha - 1}$.
\end{definition}


\subsection{Properties inherited from f-divergences}

Since $\He_\alpha$ is an f-divergence, every inequality for f-divergences can be translated to $\He_\alpha$.

\begin{theorem}[Data-processing]
  \label{thm:hellingerAlpha_data_proc}
  \lean{ProbabilityTheory.hellingerDiv_comp_right_le}
  \leanok
  \uses{def:hellingerAlpha}
  Let $\alpha > 0$, $\mu, \nu$ be two finite measures on $\mathcal X$ and let $\kappa : \mathcal X \rightsquigarrow \mathcal Y$ be a Markov kernel.
  Then $\He_\alpha(\kappa \circ \mu, \kappa \circ \nu) \le \He_\alpha(\mu, \nu)$.
\end{theorem}

\begin{proof} \leanok
\uses{thm:fDiv_data_proc_2}
Apply Theorem~\ref{thm:fDiv_data_proc_2}.
\end{proof}


\begin{lemma}
  \label{lem:hellingerAlpha_nonneg}
  \lean{ProbabilityTheory.hellingerDiv_nonneg}
  \leanok
  \uses{def:hellingerAlpha}
  Let $\mu, \nu$ be two probability measures. Then $\He_\alpha(\mu, \nu) \ge 0$.
\end{lemma}

\begin{proof} \leanok
\uses{lem:fDiv_nonneg}
Apply Lemma~\ref{lem:fDiv_nonneg}.
\end{proof}


\begin{lemma}
  \label{lem:hellingerAlpha_convex}
  %\lean{}
  %\leanok
  \uses{def:hellingerAlpha}
  $(\mu, \nu) \mapsto \He_\alpha(\mu, \nu)$ is convex.
\end{lemma}

\begin{proof}
\uses{thm:fDiv_convex}
Apply Theorem~\ref{thm:fDiv_convex}
\end{proof}



TODO: find a way to hide the following dummy lemma
\begin{lemma}[Dummy lemma: hellingerAlpha properties]
  \label{lem:hellingerAlpha_properties}
  %\lean{}
  \leanok
  \uses{def:hellingerAlpha, def:condHellingerAlpha}
  Dummy node to summarize properties of the Hellinger $\alpha$-divergence.
\end{lemma}

\begin{proof}\leanok
\uses{
  lem:hellingerAlpha_ne_top_of_lt_one,
  lem:hellingerAlpha_eq_integral,
  lem:hellingerAlpha_symm,
  thm:hellingerAlpha_data_proc,
  lem:hellingerAlpha_nonneg,
  lem:hellingerAlpha_convex
}
\end{proof}