\chapter{Hellinger alpha-divergences}

\begin{definition}[Hellinger $\alpha$-divergence]
  \label{def:hellingerAlpha}
  \lean{ProbabilityTheory.hellingerDiv}
  \leanok
  \uses{def:KL, def:fDiv}
  Let $\mu, \nu$ be two measures on $\mathcal X$. The Hellinger divergence of order $\alpha \in (0,+\infty)$ between $\mu$ and $\nu$ is
  \begin{align*}
  \He_\alpha(\mu, \nu) = \left\{
  \begin{array}{ll}
    \KL(\mu, \nu) & \text{for } \alpha = 1
    \\
    D_{f_\alpha}(\mu, \nu) & \text{for } \alpha \in (0,+\infty) \backslash \{1\}
  \end{array}\right.
  \end{align*}
  with $f_\alpha : x \mapsto \frac{x^{\alpha} - 1}{\alpha - 1}$.
\end{definition}

\begin{lemma}
  \label{lem:hellingerAlpha_symm}
  \lean{ProbabilityTheory.hellingerDiv_symm}
  \leanok
  \uses{def:hellingerAlpha}
  For $\alpha \in (0, 1)$, $(1 - \alpha) \He_\alpha(\mu, \nu) = \alpha \He_{1 - \alpha}(\nu, \mu)$.
\end{lemma}

\begin{proof}
Unfold the definitions.
\end{proof}


\begin{definition}[Conditional Hellinger $\alpha$-divergence]
  \label{def:condHellingerAlpha}
  %\lean{}
  %\leanok
  \uses{def:condFDiv}
  Let $\mu$ be a measure on $\mathcal X$ and $\kappa, \eta : \mathcal X \rightsquigarrow \mathcal Y$ be two Markov kernels. The conditional Hellinger divergence of order $\alpha \in (0,+\infty) \backslash \{1\}$ between $\kappa$ and $\eta$ conditionally to $\mu$ is
  \begin{align*}
  \He_\alpha(\kappa, \eta \mid \mu) = D_{f_\alpha}(\kappa, \eta \mid \mu) \: ,
  \end{align*}
  for $f_\alpha : x \mapsto \frac{x^{\alpha} - 1}{\alpha - 1}$.
\end{definition}


\section{Properties inherited from f-divergences}

Since $\He_\alpha$ is an f-divergence, every inequality for f-divergences can be translated to $\He_\alpha$.

\begin{theorem}[Data-processing]
  \label{thm:hellingerAlpha_data_proc}
  %\lean{}
  %\leanok
  \uses{def:hellingerAlpha}
  Let $\mu, \nu$ be two measures on $\mathcal X$ and let $\kappa : \mathcal X \rightsquigarrow \mathcal Y$ be a Markov kernel.
  Then $\He_\alpha(\kappa \circ \mu, \kappa \circ \nu) \le \He_\alpha(\mu, \nu)$.
\end{theorem}

\begin{proof}
\uses{thm:fDiv_data_proc}
Apply Theorem~\ref{thm:fDiv_data_proc}.
\end{proof}

\begin{lemma}
  \label{lem:hellingerAlpha_nonneg}
  %\lean{}
  %\leanok
  \uses{def:hellingerAlpha}
  Let $\mu, \nu$ be two probability measures. Then $\He_\alpha(\mu, \nu) \ge 0$.
\end{lemma}

\begin{proof}
\uses{lem:fDiv_nonneg}
Apply Lemma~\ref{lem:fDiv_nonneg}.
\end{proof}

\begin{lemma}
  \label{lem:hellingerAlpha_convex}
  %\lean{}
  %\leanok
  \uses{def:hellingerAlpha}
  $(\mu, \nu) \mapsto \He_\alpha(\mu, \nu)$ is convex.
\end{lemma}

\begin{proof}
\uses{thm:fDiv_convex}
Apply Theorem~\ref{thm:fDiv_convex}
\end{proof}
