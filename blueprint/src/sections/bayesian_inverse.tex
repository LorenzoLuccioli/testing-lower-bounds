\chapter{Bayesian inverse of a kernel}

\begin{definition}
  \label{def:bayesInv}
  %\lean{}
  %\leanok
  \uses{def:kernel_comp, def:measure_compProd}
  For $\mu \in \mathcal M(\mathcal X)$ and $\kappa : \mathcal X \rightsquigarrow \mathcal Y$, a Bayesian inverse of $\kappa$ is a Markov kernel $\kappa_\mu^\dagger : \mathcal Y \rightsquigarrow \mathcal X$ such that $\mu \otimes \kappa = ((\kappa \circ \mu) \otimes \kappa_\mu^\dagger)_\leftrightarrow$ in which $(\cdot)_\leftrightarrow$ denotes swapping the two coordinates.
  If such an inverse exists it is unique up to a $(\kappa \circ \mu)$-null set, and we talk about \emph{the} Bayesian inverse of $\kappa$ with respect to $\mu$.
  For $\mathcal X$ standard Borel, $\mu$ and $\kappa$ s-finite, the Bayesian inverse exists and is obtained by disintegration of the measure $\mu \otimes \kappa$ on $\mathcal X \times \mathcal Y$ into a measure $\kappa \circ \mu \in \mathcal M(\mathcal Y)$ and a Markov kernel $\kappa_\mu^\dagger: \mathcal Y \rightsquigarrow \mathcal X$.
\end{definition}

The bayesian interpretation of this definition is that $\mu$ represents a prior over a parameter and $\kappa$ gives the distribution of data given the parameter. $\kappa \circ \mu$ is the distribution of the data, and $\kappa_\mu^\dagger$ gives the posterior distribution of the parameter given the data.

See \cite{clerc2017pointless} and \cite{dahlqvist2018borel} for a category theory point of view on Bayesian inversion.

\begin{lemma}
  \label{lem:bayesInv_comp_self}
  %\lean{}
  %\leanok
  \uses{def:bayesInv}
  For $\mu \in \mathcal M(\mathcal X)$ s-finite, $\kappa : \mathcal X \rightsquigarrow \mathcal Y$ a Markov kernel and $\kappa_\mu^\dagger$ the Bayesian inverse of $\kappa$ with respect to $\mu$, these objects satisfy the equality $\kappa_\mu^\dagger \circ \kappa \circ \mu = \mu$.
\end{lemma}

\begin{proof}%\leanok
\uses{}
The measure $\kappa_\mu^\dagger \circ (\kappa \circ \mu)$ is the projection on $\mathcal X$ of $(\kappa \circ \mu) \otimes \kappa_\mu^\dagger$. But $(\kappa \circ \mu) \otimes \kappa_\mu^\dagger = (\mu \otimes \kappa)_\leftrightarrow$ by definition of the Bayesian inverse. Since $\kappa$ is a Markov kernel, the projection on $\mathcal X$ of that measure is simply $\mu$.
\end{proof}

\begin{lemma}
  \label{lem:bayesInv_self}
  %\lean{}
  %\leanok
  \uses{def:bayesInv}
  For $\mathcal X$ and $\mathcal Y$ two standard Borel spaces, $\mu \in \mathcal M(\mathcal X)$ s-finite and $\kappa : \mathcal X \rightsquigarrow \mathcal Y$ a Markov kernel, $(\kappa_\mu^\dagger)_{\kappa \circ \mu}^\dagger = \kappa$ ($\mu$-a.e.).
\end{lemma}

\begin{proof}%\leanok
\uses{lem:bayesInv_comp_self}
By uniqueness of the disintegration, it suffices to show that $(\kappa \circ \mu) \otimes \kappa_\mu^\dagger = ((\kappa_\mu^\dagger \circ \kappa \circ \mu) \otimes \kappa)_\leftrightarrow$~.
By Lemma~\ref{lem:bayesInv_comp_self}, $\kappa_\mu^\dagger \circ \kappa \circ \mu = \mu$ and we need to prove $(\kappa \circ \mu) \otimes \kappa_\mu^\dagger = (\mu \otimes \kappa)_\leftrightarrow$~.
This is true by definition of $\kappa_\mu^\dagger$~.
\end{proof}

\begin{lemma}
  \label{lem:bayesInv_id}
  %\lean{}
  %\leanok
  \uses{def:bayesInv}
  Let $\mu \in \mathcal M (\mathcal X)$ and let $\textup{id} : \mathcal X \rightsquigarrow \mathcal X$ be the identity kernel. Then $\textup{id}_\mu^\dagger = \textup{id}$.
\end{lemma}

\begin{proof}%\leanok
\uses{}
It suffices to show that $\mu \otimes \textup{id} = ((\textup{id} \circ \mu) \otimes \textup{id})_\leftrightarrow$~, which is clear.
\end{proof}

\begin{lemma}
  \label{lem:bayesInv_comp}
  %\lean{}
  %\leanok
  \uses{def:bayesInv,def:kernel_comp}
  Let $\mu \in \mathcal M(\mathcal X)$, $\kappa : \mathcal X \rightsquigarrow \mathcal Y$ and $\eta : \mathcal Y \rightsquigarrow \mathcal Z$. Then $(\eta \circ \kappa \circ \mu)$-a.e.,
  \begin{align*}
  (\eta \circ \kappa)_\mu^\dagger = \kappa_{\mu}^\dagger \circ \eta_{\kappa \circ \mu}^\dagger
  \: .
  \end{align*}
\end{lemma}

\begin{proof}%\leanok
\uses{}
It suffices to show that $\mu \otimes (\eta \circ \kappa) = ((\eta \circ \kappa \circ \mu) \otimes (\kappa_{\mu}^\dagger \circ \eta_{\kappa \circ \mu}^\dagger))_\leftrightarrow$.

TODO:
\begin{align*}
\mu \otimes (\eta \circ \kappa)
&= (\mu \otimes \kappa \otimes \eta)_{XZ}
\\
&= (((\kappa \circ \mu) \otimes \kappa_\mu^\dagger)_{\leftrightarrow} \otimes \eta)_{XZ}
\\
&= ... ?
\end{align*}

\begin{align*}
(\eta \circ \kappa \circ \mu) \otimes (\kappa_{\mu}^\dagger \circ \eta_{\kappa \circ \mu}^\dagger)
&= ((\eta \circ \kappa \circ \mu) \otimes \eta_{\kappa \circ \mu}^\dagger \otimes \kappa_{\mu}^\dagger)_{ZX}
\\
&= (((\kappa \circ \mu) \otimes \eta)_\leftrightarrow \otimes \kappa_{\mu}^\dagger)_{ZX}
\\
&= ... ?
\end{align*}

\end{proof}