\section{Rényi divergences}

\begin{definition}[Rényi divergence]
  \label{def:Renyi}
  \lean{ProbabilityTheory.renyiDiv}
  \leanok
  \uses{def:KL, def:hellingerAlpha, lem:hellingerAlpha_properties}
  Let $\mu, \nu$ be two measures on $\mathcal X$. The Rényi divergence of order $\alpha \in \mathbb{R}$ between $\mu$ and $\nu$ is
  \begin{align*}
  R_\alpha(\mu, \nu) = \left\{
 \begin{array}{ll}
  \KL(\mu, \nu) & \text{for } \alpha = 1
  \\
  \frac{1}{\alpha - 1} \log (\nu(\mathcal X)  + (\alpha - 1) \He_\alpha(\mu, \nu)) & \text{for } \alpha \neq 1
  \end{array}\right.
  \end{align*}
\end{definition}


\begin{lemma}
  \label{lem:renyiDiv_zero}
  \lean{ProbabilityTheory.renyiDiv_zero}
  \leanok
  \uses{def:Renyi}
  For $\mu$ a sigma-finite measure and $\nu$ a finite measure 
  \[R_0(\mu, \nu) = - \log(\nu\{x \mid \frac{d \mu}{d \nu}(x) > 0\}) \: .\]
\end{lemma}

\begin{proof}\leanok
\uses{def:hellingerAlpha}
\end{proof}


\begin{lemma}
  \label{lem:renyiDiv_eq_top_iff_mutuallySingular_of_lt_one}
  \lean{ProbabilityTheory.renyiDiv_eq_top_iff_mutuallySingular_of_lt_one}
  \leanok
  \uses{def:Renyi}
  For $\alpha \in [0, 1)$ and finite measures $\mu, \nu$, 
  \[R_\alpha(\mu, \nu) = \infty \iff \mu \perp \nu \: .\]
\end{lemma}

\begin{proof}\leanok
\uses{lem:hellingerAlpha_ne_top_of_lt_one}
\end{proof}


\begin{lemma}
  \label{lem:renyi_eq_log_integral}
  \lean{ProbabilityTheory.renyiDiv_eq_log_integral}
  \leanok
  \uses{def:Renyi}
  For $\alpha \in (0,1)\cup(1, \infty)$ and finite measures $\mu, \nu$, if 
  $\left(\frac{d \mu}{d \nu}\right)^\alpha$ is integrable with respect to $\nu$ and $\mu \ll \nu$ then
  \begin{align*}
  R_\alpha(\mu, \nu) = \frac{1}{\alpha - 1} \log \int_x \left(\frac{d \mu}{d \nu}(x)\right)^\alpha \partial \nu
  \: .
  \end{align*}
\end{lemma}

\begin{proof}\leanok
\uses{lem:hellingerAlpha_eq_integral}
\end{proof}


\begin{lemma}
  \label{lem:renyi_eq_log_integral'}
  \lean{ProbabilityTheory.renyiDiv_eq_log_integral'}
  \leanok
  \uses{def:Renyi}
  For $\alpha \in (0,1)\cup(1, \infty)$ and finite measures $\mu, \nu$, if 
  $\left(\frac{d \mu}{d \nu}\right)^\alpha$ is integrable with respect to $\nu$ and $\mu \ll \nu$ then
  \begin{align*}
  R_\alpha(\mu, \nu) = \frac{1}{\alpha - 1} \log \int_x \left(\frac{d \mu}{d \nu}(x)\right)^{\alpha - 1} \partial \mu
  \: .
  \end{align*}
\end{lemma}

\begin{proof}\leanok
\uses{lem:renyi_eq_log_integral, lem:integral_rpow_rnDeriv}
\end{proof}


\begin{lemma}
  \label{lem:renyi_symm}
  \lean{ProbabilityTheory.renyiDiv_symm'}
  \leanok
  \uses{def:Renyi}
  For $\alpha \in (0, 1)$ and finite measures $\mu, \nu$ with $\mu(\mathcal X) = \nu(\mathcal X)$, 
  \[(1 - \alpha) R_\alpha(\mu, \nu) = \alpha R_{1 - \alpha}(\nu, \mu) \: .\]
\end{lemma}

\begin{proof}\leanok
\uses{lem:hellingerAlpha_symm}
Use Lemma~\ref{lem:hellingerAlpha_symm}.
\end{proof}


\begin{lemma}
  \label{lem:renyi_cgf}
  \lean{ProbabilityTheory.cgf_llr_of_lt_one}
  \leanok
  \uses{def:Renyi}
  The cumulant generating function of $\log\frac{d\mu}{d\nu}$ under $\nu$ is $\alpha \mapsto (\alpha - 1) R_\alpha(\mu, \nu)$.
\end{lemma}

\begin{proof}\leanok
\uses{lem:renyi_eq_log_integral}
Unfold the definitions, using Lemma~\ref{lem:renyi_eq_log_integral} for the Rényi divergence.
\end{proof}


\begin{lemma}
  \label{lem:renyi_cgf_2}
  \lean{ProbabilityTheory.cgf_llr'}
  \leanok
  \uses{def:Renyi}
  Set $\alpha > 0$. If $R_{1+\alpha}(\mu, \nu) < \infty$, the cumulant generating function of $\log\frac{d\mu}{d\nu}$ under $\mu$ has value $\alpha R_{1+\alpha}(\mu, \nu)$ at $\alpha$.
\end{lemma}

\begin{proof}\leanok
\uses{lem:renyi_eq_log_integral'}
Unfold the definitions, using Lemma~\ref{lem:renyi_eq_log_integral'} for the Rényi divergence.
\end{proof}


\begin{definition}[Conditional Rényi divergence]
  \label{def:condRenyi}
  \lean{ProbabilityTheory.condRenyiDiv}
  \leanok
  \uses{def:condHellingerAlpha, lem:hellingerAlpha_properties}
  Let $\mu$ be a measure on $\mathcal X$ and $\kappa, \eta : \mathcal X \rightsquigarrow \mathcal Y$ be two kernels. The conditional Rényi divergence of order $\alpha \in (0,+\infty) \backslash \{1\}$ between $\kappa$ and $\eta$ conditionally to $\mu$ is
  \begin{align*}
  R_\alpha(\kappa, \eta \mid \mu) =\frac{1}{\alpha - 1} \log (1 + (\alpha - 1) \He_\alpha(\kappa, \eta \mid \mu)) \: .
  \end{align*}
\end{definition}

TODO: the reasons for this definition is that it is equal to $R_\alpha(\mu \otimes \kappa, \mu \otimes \eta)$, which is the generic conditional divergence definition, and also that this is the definition for which we have a chain rule.




\subsection{Properties inherited from f-divergences}

Since Rényi divergences are monotone transfomations of f-divergences, every inequality for f-divergences can be translated to Rényi divergences.

\begin{theorem}[Data-processing]
  \label{thm:renyi_data_proc}
  \lean{ProbabilityTheory.renyiDiv_comp_right_le}
  \leanok
  \uses{def:Renyi}
  Let $\mu, \nu$ be two finite measures on $\mathcal X$ and let $\kappa : \mathcal X \rightsquigarrow \mathcal Y$ be a Markov kernel.
  Then $R_\alpha(\kappa \circ \mu, \kappa \circ \nu) \le R_\alpha(\mu, \nu)$.
\end{theorem}

\begin{proof} \leanok
\uses{thm:fDiv_data_proc_2}
The function $x \mapsto \frac{1}{\alpha - 1}\log (1 + (\alpha - 1)x)$ is non-decreasing and $D_f$ satisfies the DPI (Theorem~\ref{thm:fDiv_data_proc_2}), hence we get the DPI for $R_\alpha$.
\end{proof}

\begin{lemma}
  \label{lem:renyi_data_proc_event}
  %\lean{}
  %\leanok
  \uses{def:Renyi}
  Let $\mu, \nu$ be two measures on $\mathcal X$ and let $E$ be an event. Let $\mu_E$ and $\nu_E$ be the two Bernoulli distributions with respective means $\mu(E)$ and $\nu(E)$.
  Then $R_\alpha(\mu, \nu) \ge R_\alpha(\mu_E, \nu_E)$.
\end{lemma}

\begin{proof}
\uses{cor:data_proc_event}
By Corollary~\ref{cor:data_proc_event}, $D_f(\mu, \nu) \ge D_f(\mu_E, \nu_E)$, hence $R_\alpha(\mu, \nu) \ge R_\alpha(\mu_E, \nu_E)$.
\end{proof}




\subsection{Comparisons between orders}

\begin{lemma}
  \label{lem:renyi_tendsto_renyi_zero}
  %\lean{}
  %\leanok
  \uses{def:Renyi}
  Let $\mu, \nu$ be two finite measures. $R_0(\mu, \nu) = \lim_{\alpha \downarrow 0} R_\alpha(\mu, \nu)$.
\end{lemma}

\begin{proof}
\end{proof}

\begin{lemma}
  \label{lem:renyi_tendsto_renyi_one}
  %\lean{}
  %\leanok
  \uses{def:Renyi}
  Let $\mu, \nu$ be two finite measures. $R_1(\mu, \nu) = \lim_{\alpha \uparrow 1} R_\alpha(\mu, \nu)$.
\end{lemma}

\begin{proof}
\end{proof}

\begin{lemma}
  \label{lem:renyi_tendsto_renyi_one_above}
  %\lean{}
  %\leanok
  \uses{def:Renyi}
  Let $\mu, \nu$ be two finite measures such that there exists $\alpha > 1$ with $R_\alpha(\mu, \nu)$ finite. Then $R_1(\mu, \nu) = \lim_{\alpha \downarrow 1} R_\alpha(\mu, \nu)$.
\end{lemma}

\begin{proof}
\end{proof}

\begin{lemma}
  \label{lem:renyi_monotone}
  %\lean{}
  %\leanok
  \uses{def:Renyi}
  Let $\mu, \nu$ be two finite measures. Then $\alpha \mapsto R_\alpha(\mu, \nu)$ is nondecreasing on $[0, + \infty)$.
\end{lemma}

\begin{proof}
\end{proof}

\begin{lemma}
  \label{lem:renyi_continuous}
  %\lean{}
  %\leanok
  \uses{def:Renyi}
  Let $\mu, \nu$ be two finite measures. Then $\alpha \mapsto R_\alpha(\mu, \nu)$ is continuous on $[0, 1]$ and on $[0, \sup \{\alpha \mid R_\alpha(\mu, \nu) < \infty\})$.
\end{lemma}

\begin{proof}
\end{proof}




\subsection{The tilted measure}

\begin{definition}
  \label{def:renyiMeasure}
  \lean{ProbabilityTheory.renyiMeasure}
  \leanok
  \uses{def:Renyi}
  Let $\mu, \nu$ be two measures on $\mathcal X$ and let $\alpha \in (0, +\infty) \backslash \{1\}$. Let $p = \frac{d \mu}{d (\mu + \nu)}$ and $q = \frac{d \nu}{d (\mu + \nu)}$. We define a measure $\mu^{(\alpha, \nu)}$, absolutely continuous with respect to $\mu + \nu$ with density
  \begin{align*}
  \frac{d \mu^{(\alpha, \nu)}}{d (\mu + \nu)} = p^\alpha q^{1 - \alpha} e^{-(\alpha - 1)R_\alpha(\mu, \nu)} \: .
  \end{align*}
\end{definition}


\begin{lemma}
  \label{lem:renyiMeasure_ac}
  %\lean{}
  %\leanok
  \uses{def:renyiMeasure}
  For $\alpha \in (0,1)$, $\mu^{(\alpha, \nu)} \ll \mu$ and $\mu^{(\alpha, \nu)} \ll \nu$.
\end{lemma}

\begin{proof}%\leanok
\uses{}

\end{proof}


\begin{lemma}
  \label{lem:rnDeriv_renyiMeasure}
  %\lean{}
  %\leanok
  \uses{def:renyiMeasure}
  $\frac{d \mu^{(\alpha, \nu)}}{d \nu} = \left(\frac{d\mu}{d\nu}\right)^\alpha e^{-(\alpha - 1) R_\alpha(\mu, \nu)}$~, $\nu$-a.e.,
  and
  $\frac{d \mu^{(\alpha, \nu)}}{d \mu} = \left(\frac{d\nu}{d\mu}\right)^{1 - \alpha} e^{-(\alpha - 1) R_\alpha(\mu, \nu)}$~, $\mu$-a.e..
\end{lemma}

\begin{proof}%\leanok
\uses{lem:renyiMeasure_ac}

\end{proof}


\begin{lemma}
  \label{lem:kl_renyiMeasure_eq}
  %\lean{}
  %\leanok
  \uses{def:renyiMeasure, def:Renyi, def:KL}
  Let $\mu, \nu, \xi$ be three measures on $\mathcal X$ and let $\alpha \in (0, 1)$. Then
  \begin{align*}
  \KL(\xi, \mu^{(\alpha, \nu)}) = \alpha \KL(\xi, \mu) + (1 - \alpha)\KL(\xi, \nu) - (1 - \alpha) R_\alpha(\mu, \nu) \: .
  \end{align*}
\end{lemma}

\begin{proof}
\uses{def:renyiMeasure, def:KL}
Unfold definitions and compute?
\end{proof}


\begin{corollary}
  \label{cor:renyi_eq_add_kl}
  %\lean{}
  %\leanok
  \uses{def:renyiMeasure, def:Renyi, def:KL}
  Let $\mu, \nu, \xi$ be three measures on $\mathcal X$ and let $\alpha \in (0, 1)$. Then
  \begin{align*}
  (1 - \alpha) R_\alpha(\mu, \nu) = \alpha \KL(\mu^{(\alpha, \nu)}, \mu) + (1 - \alpha)\KL(\mu^{(\alpha, \nu)}, \nu) \: .
  \end{align*}
\end{corollary}

\begin{proof}
\uses{lem:kl_renyiMeasure_eq}
Apply Lemma~\ref{lem:kl_renyiMeasure_eq} to $\xi = \mu^{(\alpha, \nu)}$.
\end{proof}


\begin{lemma}
  \label{lem:renyi_eq_inf_add_kl}
  %\lean{}
  %\leanok
  \uses{def:Renyi, def:KL}
  Let $\mu, \nu$ be two probability measures on $\mathcal X$ and let $\alpha \in (0, 1)$. Then
  \begin{align*}
  (1 - \alpha) R_\alpha(\mu, \nu) = \inf_{\xi \in \mathcal P(\mathcal X)}\left( \alpha \KL(\xi, \mu) + (1 - \alpha)\KL(\xi, \nu) \right) \: .
  \end{align*}
  The infimum is attained at $\xi = \mu^{(\alpha, \nu)}$.
\end{lemma}

\begin{proof}
\uses{lem:kl_renyiMeasure_eq, lem:kl_nonneg, cor:renyi_eq_add_kl}
By Lemma~\ref{lem:kl_renyiMeasure_eq} and Lemma~\ref{lem:kl_nonneg}, for all $\xi$,
\begin{align*}
\alpha \KL(\xi, \mu) + (1 - \alpha)\KL(\xi, \nu) - (1 - \alpha) R_\alpha(\mu, \nu)
= \KL(\xi, \mu^{(\alpha, \nu)})
\ge 0 \: .
\end{align*}
We have then $(1 - \alpha) R_\alpha(\mu, \nu) \le \inf_{\xi \in \mathcal P(\mathcal X)}\left( \alpha \KL(\xi, \mu) + (1 - \alpha)\KL(\xi, \nu) \right)$.
Corollary~\ref{cor:renyi_eq_add_kl} proves the other inequality.
\end{proof}


\begin{lemma}
  \label{lem:renyi_eq_inf_kl}
  %\lean{}
  %\leanok
  \uses{def:Renyi, def:KL}
  Let $\mu, \nu \in \mathcal P(\mathcal X)$ and let $\alpha \in (0, 1)$. Let $\pi_\alpha = (\alpha, 1 - \alpha) \in \mathcal P(\{0,1\})$ and let $P : \{0,1\} \rightsquigarrow \mathcal X$ be the kernel with $P(0) = \mu$ and $P(1) = \nu$. Then
  \begin{align*}
  (1 - \alpha) R_\alpha(\mu, \nu) = \inf_{\xi \in \mathcal P(\mathcal X)} \KL\left( \pi_\alpha \times \xi, \pi_\alpha \otimes P \right) \: .
  \end{align*}
  The infimum is attained at $\xi = \mu^{(\alpha, \nu)}$.
\end{lemma}

Compare with Lemma~\ref{lem:jensenShannon_eq_inf_kl} on the Jensen-Shannon divergence.

\begin{proof}%\leanok
\uses{lem:renyi_eq_inf_add_kl}
\begin{align*}
\KL\left( \pi_\alpha \times \xi, \pi_\alpha \otimes P \right)
&= \alpha \KL(\xi, \mu) + (1 - \alpha) \KL(\xi, \nu)
\: .
\end{align*}
The result is a rephrasing of Lemma~\ref{lem:renyi_eq_inf_add_kl}.
\end{proof}



\subsection{Chain rule and tensorization}

\begin{theorem}[Chain rule]
  \label{thm:renyi_chain_rule}
  %\lean{}
  %\leanok
  \uses{def:renyiMeasure, def:condRenyi}
  Let $\mu, \nu$ be two measures on $\mathcal X$ and let $\kappa, \eta : \mathcal X \rightsquigarrow \mathcal Y$ be two Markov kernels.
  Then $R_\alpha(\mu \otimes \kappa, \nu \otimes \eta) = R_\alpha(\mu, \nu) + R_\alpha(\kappa, \eta \mid \mu^{(\alpha, \nu)})$.
\end{theorem}

\begin{proof}%\leanok
\uses{thm:kl_compProd}
\end{proof}


\begin{theorem}[Chain rule, with Bayesian inverse]
  \label{thm:renyi_compProd_bayesInv}
  %\lean{}
  %\leanok
  \uses{def:Renyi, def:condRenyi, def:bayesInv}
  Let $\mu, \nu$ be two finite measures on $\mathcal X$ and $\kappa, \eta : \mathcal X \rightsquigarrow \mathcal Y$ two Markov kernels with Bayesian inverses $\kappa_\mu^\dagger$ and $\eta_\nu^\dagger$.
  Then 
  \begin{align*}
  R_\alpha(\mu \otimes \kappa, \nu \otimes \eta)
  = R_\alpha(\kappa \circ \mu, \eta \circ \nu) + R_\alpha(\kappa_\mu^\dagger, \eta_\nu^\dagger \mid (\kappa \circ \mu)^{(\alpha, \eta \circ \nu)})
  \: .
  \end{align*}
\end{theorem}

\begin{proof}%\leanok
\uses{thm:renyi_chain_rule}

\end{proof}


\begin{corollary}
  \label{cor:renyi_prod_two}
  %\lean{}
  %\leanok
  \uses{def:Renyi}
  Let $\mu, \nu \in \mathcal P(\mathcal X)$ and $\xi, \lambda \in \mathcal P(\mathcal Y)$.
  Then $R_\alpha(\mu \times \xi, \nu \times \lambda) = R_\alpha(\mu, \nu) + R_\alpha(\xi, \lambda)$.
\end{corollary}

\begin{proof}%\leanok
\uses{thm:renyi_chain_rule}
Apply Theorem~\ref{thm:renyi_chain_rule} (the product of measures is a special case of $\otimes$). TODO: more details.
\end{proof}


\begin{theorem}[Tensorization - finite product]
  \label{thm:renyi_prod}
  %\lean{}
  %\leanok
  \uses{def:Renyi}
  Let $I$ be a finite index set. Let $(\mu_i)_{i \in I}, (\nu_i)_{i \in I}$ be probability measures on measurable spaces $(\mathcal X_i)_{i \in I}$.
  Then $R_\alpha (\prod_{i \in I} \mu_i, \prod_{i \in I} \nu_i) = \sum_{i \in I} R_\alpha(\mu_i, \nu_i)$.
\end{theorem}

\begin{proof}%\leanok
\uses{cor:renyi_prod_two}
Induction over the finite index set, using Corollary~\ref{cor:renyi_prod_two}.
\end{proof}


\begin{corollary}
  \label{lem:renyi_prod_n}
  %\lean{}
  %\leanok
  \uses{def:Renyi}
  Let $\mu, \nu$ be two probability measures on $\mathcal X$. Let $n \in \mathbb{N}$ and write $\mu^{\otimes n}$ for the product measure on $\mathcal X^n$ of $n$ times $\mu$.
  Then $R_\alpha(\mu^{\otimes n}, \nu^{\otimes n}) = n R_\alpha(\mu, \nu)$.
\end{corollary}

\begin{proof}%\leanok
\uses{thm:renyi_prod}
Apply Theorem~\ref{thm:renyi_prod}.
\end{proof}


\begin{theorem}[Tensorization - countable product]
  \label{thm:renyi_prod_countable}
  %\lean{}
  %\leanok
  \uses{def:Renyi}
  Let $I$ be a countable index set. Let $(\mu_i)_{i \in I}, (\nu_i)_{i \in I}$ be probability measures on measurable spaces $(\mathcal X_i)_{i \in I}$.
  Then $R_\alpha (\prod_{i \in I} \mu_i, \prod_{i \in I} \nu_i) = \sum_{i \in I} R_\alpha(\mu_i, \nu_i)$.
\end{theorem}

\begin{proof}%\leanok
\uses{thm:renyi_prod}
\end{proof}
