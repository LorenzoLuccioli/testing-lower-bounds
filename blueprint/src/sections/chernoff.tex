\section{Chernoff divergence}

\begin{definition}[Chernoff divergence]
  \label{def:Chernoff}
  \lean{ProbabilityTheory.chernoffDiv}
  \leanok
  \uses{def:Renyi}
    The Chernoff divergence of order $\alpha > 0$ between two measures $\mu$ and $\nu$ on $\mathcal X$ is
  \begin{align*}
    C_\alpha(\mu, \nu) = \inf_{\xi \in \mathcal P(\mathcal X)}\max\{R_\alpha(\xi, \mu), R_\alpha(\xi, \nu)\} \: .
  \end{align*}
\end{definition}

Note: the name Chernoff divergence is usually reserved for $C_1$, and any reference to $C$ without subscript in this document refers to $C_1$.
The extension to other orders $\alpha$ is motivated by the appearance of $C_\alpha$ in a result further down.
This is not a well established notion and it remains to see if this extension is meaningful.

\begin{lemma}
  \label{lem:chernoff_eq_kl}
  \lean{ProbabilityTheory.chernoffDiv_one}
  \leanok
  \uses{def:Chernoff, def:KL}
  $C_1(\mu, \nu) = \inf_{\xi \in \mathcal P(\mathcal X)}\max\{\KL(\xi, \mu), \KL(\xi, \nu)\}$ .
\end{lemma}

\begin{proof}\leanok
\uses{def:Renyi}
This is $R_1 = \KL$ (by definition).
\end{proof}

\begin{lemma}[Symmetry]
  \label{lem:chernoff_symm}
  %\lean{}
  %\leanok
  \uses{def:Chernoff}
  $C_\alpha(\mu, \nu) = C_\alpha(\nu, \mu)$.
\end{lemma}

\begin{proof}%\leanok
Immediate from the definition.
\end{proof}

\begin{lemma}[Monotonicity]
  \label{lem:chernoff_mono}
  %\lean{}
  %\leanok
  \uses{def:Chernoff}
  The function $\alpha \mapsto C_\alpha(\mu, \nu)$ is monotone.
\end{lemma}

\begin{proof}
\uses{lem:renyi_monotone}
Consequence of Lemma~\ref{lem:renyi_monotone}.
\end{proof}

\begin{lemma}
  \label{lem:chernoff_eq_max_renyi}
  %\lean{}
  %\leanok
  \uses{def:Chernoff, def:Renyi}
  \begin{align*}
  C_1(\mu, \nu) = \max_{\alpha\in [0,1]} (1 - \alpha)R_\alpha(\mu, \nu) \: .
  \end{align*}
\end{lemma}

\begin{proof}%\leanok
\uses{}
\end{proof}

