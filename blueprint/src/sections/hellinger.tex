\chapter{Hellinger distance}

\begin{definition}[Squared Hellinger distance]
  \label{def:Hellinger}
  %\lean{}
  %\leanok
  \uses{def:fDiv}
  Let $\mu, \nu$ be two measures. The squared Hellinger distance between $\mu$ and $\nu$ is
  \begin{align*}
    H^2(\mu, \nu) = D_f(\mu, \nu) \quad \text{with } f: x \mapsto \frac{1}{2}\left( 1 - \sqrt{x} \right)^2 \: .
  \end{align*}
\end{definition}

\begin{lemma}[Hellinger and Rényi]
  \label{lem:renyi_half_eq_log_hellinger}
  %\lean{}
  %\leanok
  \uses{def:Hellinger, def:Renyi}
  Let $\mu, \nu$ be two probability measures. Then $R_{1/2}(\mu, \nu) = -2\log(1 - H^2(\mu, \nu))$.
\end{lemma}

\begin{proof}
\uses{lem:renyi_eq_log_fDiv, lem:fDiv_mul, lem:fDiv_add_linear}
By Lemma~\ref{lem:renyi_eq_log_fDiv}, $R_{1/2}(\mu, \nu) = -2 \log (1 - \frac{1}{2} D_f(\nu, \mu))$ for $f : x \mapsto -2 (\sqrt{x} - 1)$. Using Lemma~\ref{lem:fDiv_mul}, $R_{1/2}(\mu, \nu) = -2 \log (1 - D_g(\nu, \mu))$ for $g(x) = 1 - \sqrt{x}$.

It suffices then to show that $H^2(\mu, \nu) = D_g(\mu, \nu)$, which is true by an application of Lemma~\ref{lem:fDiv_add_linear}.
\end{proof}

\begin{lemma}
  \label{lem:hellinger_le_tv}
  %\lean{}
  %\leanok
  \uses{def:Hellinger, def:TV}
  Let $\mu, \nu$ be two probability measures. Then $H^2(\mu, \nu) \le TV(\mu, \nu)$.
\end{lemma}

\begin{proof}
\end{proof}

\begin{lemma}
  \label{lem:tv_le_hellinger}
  %\lean{}
  %\leanok
  \uses{def:Hellinger, def:TV}
  Let $\mu, \nu$ be two probability measures. Then $TV(\mu, \nu) \le \sqrt{H^2(\mu, \nu)(2 - H^2(\mu, \nu))}$.
\end{lemma}

\begin{proof}
\end{proof}

\begin{corollary}
  \label{cor:one_sub_hellinger_squared_le_one_sub_tv}
  %\lean{}
  %\leanok
  \uses{def:Hellinger, def:TV}
  Let $\mu, \nu$ be two probability measures. Then $\frac{1}{2}(1 - H^2(\mu, \nu))^2 \le 1 - TV(\mu, \nu)$.
\end{corollary}

\begin{proof}
\uses{lem:tv_le_hellinger}
We use Lemma~\ref{lem:tv_le_hellinger}.
\begin{align*}
1 - TV(\mu, \nu) - \frac{1}{2}(1 - H^2(\mu, \nu))^2
&= \frac{1}{2} - TV(\mu, \nu) + \frac{1}{2}(H^2(\mu, \nu)(2 - H^2(\mu, \nu)))
\\
&\ge \frac{1}{2} - TV(\mu, \nu) + \frac{1}{2}TV^2(\mu, \nu)
\\
&= \frac{1}{2}(1 - TV(\mu, \nu))^2
\\
&\ge 0
\: .
\end{align*}
\end{proof}
