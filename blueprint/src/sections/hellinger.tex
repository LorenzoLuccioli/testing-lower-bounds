\section{Hellinger distance}

TODO: refactor this section to link it to the Hellinger $\alpha$-divergence.

\begin{definition}[Squared Hellinger distance]
  \label{def:Hellinger}
  \lean{ProbabilityTheory.sqHellinger}
  \leanok
  \uses{def:fDiv}
  Let $\mu, \nu$ be two measures. The squared Hellinger distance between $\mu$ and $\nu$ is
  \begin{align*}
    \Hsq(\mu, \nu) = D_f(\mu, \nu) \quad \text{with } f: x \mapsto \frac{1}{2}\left( 1 - \sqrt{x} \right)^2 \: .
  \end{align*}
\end{definition}

\begin{lemma}[Hellinger and Rényi]
  \label{lem:renyi_half_eq_log_hellinger}
  %\lean{}
  %\leanok
  \uses{def:Hellinger, def:Renyi}
  Let $\mu, \nu$ be two probability measures. Then $R_{1/2}(\mu, \nu) = -2\log(1 - \Hsq(\mu, \nu))$.
\end{lemma}

\begin{proof}
\uses{lem:fDiv_mul, lem:fDiv_add_linear}
$R_{1/2}(\mu, \nu) = -2 \log (1 - \frac{1}{2} D_f(\nu, \mu))$ for $f : x \mapsto -2 (\sqrt{x} - 1)$. Using Lemma~\ref{lem:fDiv_mul}, $R_{1/2}(\mu, \nu) = -2 \log (1 - D_g(\nu, \mu))$ for $g(x) = 1 - \sqrt{x}$.

It suffices then to show that $\Hsq(\mu, \nu) = D_g(\mu, \nu)$, which is true by an application of Lemma~\ref{lem:fDiv_add_linear}.
\end{proof}

\begin{lemma}
  \label{lem:hellinger_le_renyi}
  %\lean{}
  %\leanok
  \uses{def:Renyi, def:Hellinger}
  Let $\mu, \nu$ be two probability measures. Then $2 \Hsq(\mu, \nu) \le R_{1/2}(\mu, \nu)$.
\end{lemma}

\begin{proof}
\uses{lem:renyi_half_eq_log_hellinger}
Use Lemma~\ref{lem:renyi_half_eq_log_hellinger}, then $-\log(1 - x) \ge x$.
\end{proof}

\begin{lemma}
  \label{lem:hellinger_le_tv}
  %\lean{}
  %\leanok
  \uses{def:Hellinger, def:TV}
  Let $\mu, \nu$ be two probability measures. Then $\Hsq(\mu, \nu) \le \TV(\mu, \nu)$.
\end{lemma}

\begin{proof}
\end{proof}

\begin{lemma}
  \label{lem:tv_le_hellinger}
  %\lean{}
  %\leanok
  \uses{def:Hellinger, def:TV}
  Let $\mu, \nu$ be two probability measures. Then $\TV(\mu, \nu) \le \sqrt{\Hsq(\mu, \nu)(2 - \Hsq(\mu, \nu))}$.
\end{lemma}

\begin{proof}
\end{proof}

\begin{corollary}
  \label{cor:one_sub_hellinger_squared_le_one_sub_tv}
  %\lean{}
  %\leanok
  \uses{def:Hellinger, def:TV}
  Let $\mu, \nu$ be two probability measures. Then
  \begin{align*}
  \frac{1}{2}(1 - \Hsq(\mu, \nu))^2 \le 1 - \TV(\mu, \nu) - \frac{1}{2}(1 - \TV(\mu, \nu))^2
  \: .
  \end{align*}
\end{corollary}

\begin{proof}
\uses{lem:tv_le_hellinger}
We use Lemma~\ref{lem:tv_le_hellinger}.
\begin{align*}
1 - \TV(\mu, \nu) - \frac{1}{2}(1 - \Hsq(\mu, \nu))^2
&= \frac{1}{2} - \TV(\mu, \nu) + \frac{1}{2}(\Hsq(\mu, \nu)(2 - \Hsq(\mu, \nu)))
\\
&\ge \frac{1}{2} - \TV(\mu, \nu) + \frac{1}{2}\TV^2(\mu, \nu)
\\
&= \frac{1}{2}(1 - \TV(\mu, \nu))^2
\: .
\end{align*}
\end{proof}

\begin{corollary}
  \label{cor:one_sub_hellinger_le_one_sub_tv}
  %\lean{}
  %\leanok
  \uses{def:Hellinger, def:TV}
  Let $\mu, \nu$ be two probability measures. Then
  \begin{align*}
  1 - \sqrt{1 - (1 - \Hsq(\mu, \nu))^2} \le 1 - \TV(\mu, \nu) \le 1 - \Hsq(\mu, \nu)
  \: .
  \end{align*}
\end{corollary}

\begin{proof}
\uses{lem:tv_le_hellinger, lem:hellinger_le_tv}
Use Lemma~\ref{lem:tv_le_hellinger} and~\ref{lem:hellinger_le_tv}.
\end{proof}

\begin{corollary}
  \label{cor:one_sub_tv_bound_renyi}
  %\lean{}
  %\leanok
  \uses{def:Renyi, def:TV}
  Let $\mu, \nu$ be two probability measures on $\mathcal X$ and let $n \in \mathbb{N}$, and $\mu^{\otimes n}, \nu^{\otimes n}$ be their product measures on $\mathcal X^n$. Then
  \begin{align*}
  \frac{1}{2}e^{-n R_{1/2}(\mu, \nu)} \le 1 - \sqrt{1 - e^{-n R_{1/2}(\mu, \nu)}} \le 1 - \TV(\mu^{\otimes n}, \nu^{\otimes n}) \le e^{-\frac{1}{2} n R_{1/2}(\mu, \nu)}
  \: .
  \end{align*}
\end{corollary}

\begin{proof}
\uses{cor:one_sub_hellinger_le_one_sub_tv, lem:renyi_half_eq_log_hellinger, lem:renyi_prod_n}
Use Corollary~\ref{cor:one_sub_hellinger_le_one_sub_tv} and Lemma~\ref{lem:renyi_half_eq_log_hellinger} to get to a comparison of $\TV(\mu, \nu)$ and $R_{1/2}(\mu^{\otimes n}, \nu^{\otimes n})$. Finally $R_{1/2}(\mu^{\otimes n}, \nu^{\otimes n}) = n R_{1/2}(\mu, \nu)$ by Lemma~\ref{lem:renyi_prod_n}.
\end{proof}
